% ----------------------------------------------
% abnTeX2: Modelo de Trabalho Academico (tese de doutorado, dissertacao de mestrado e trabalhos monograficos em geral) em conformidade com ABNT NBR 14724:2011: Informacao e documentacao - Trabalhos academicos - Apresentacao
% ----------------------------------------------

% ==============================================
% ||||||||||||||||||||||||||||||||||||||||||||||
% ----------------------------------------------

\documentclass[
	% -- opções da classe memoir --
	12pt,		% tamanho da fonte
	%openright,	% capítulos começam em pág ímpar (insere página vazia caso preciso)
	oneside,	% para impressão em verso e anverso. Oposto a oneside
	a4paper,	% tamanho do papel. 
	% -- opções da classe abntex2 --
	chapter=TITLE,		% títulos de capítulos convertidos em letras maiúsculas
	%section=TITLE,		% títulos de seções convertidos em letras maiúsculas
	%subsection=TITLE,	% títulos de subseções convertidos em letras maiúsculas
	%subsubsection=TITLE,% títulos de subsubseções convertidos em letras maiúsculas
	% -- opções do pacote babel --
	english,	% idioma adicional para hifenização
%	french,		% idioma adicional para hifenização - problemas com circuitikz
%	spanish,	% idioma adicional para hifenização
	brazil		% o último idioma é o principal do documento
	]{abntex2}

% ----------------------------------------------
% Pacotes de fontes... 
% ----------------------------------------------
\usepackage[utf8]{inputenc}	% Codificacao do documento (conversão automática dos acentos)
\usepackage[T1]{fontenc}	% Selecao de codigos de fonte. Afeta separação de sílabas
\usepackage{lmodern}	% Usa a fonte Latin Modern
\renewcommand{\ABNTEXchapterfont}{\fontfamily{ptm}\fontseries{sbc}\selectfont} % Familia de fontes times
%\usepackage{times}		% Usa a fonte Times
%\usepackage{palatino}	% Usa a fonte Palatino
%\usepackage{mathpazo}	% Usa a fonte Adobe Palatino
%\usepackage[scaled=.92]{helvet}	% Usa a fonte Helvetica
\usepackage{mathptmx}	% para utilização de times

% ----------------------------------------------
% Configuração das fontes
% ----------------------------------------------
% Algumas configurações de fontes para capitulos e seções tanto no texto quanto no sumário
\renewcommand{\ABNTEXchapterfont}{\bfseries}
\renewcommand{\ABNTEXchapterfontsize}{\Large}
\renewcommand{\ABNTEXpartfont}{\ABNTEXchapterfont}
\renewcommand{\ABNTEXpartfontsize}{\ABNTEXchapterfontsize}
\renewcommand{\cftpartfont}{\normalfont\bfseries}
\renewcommand{\ABNTEXsectionfont}{\bfseries}
\renewcommand{\ABNTEXsectionfontsize}{\large}
\renewcommand{\ABNTEXsubsectionfont}{\normalfont}
\renewcommand{\ABNTEXsubsectionfontsize}{\normalsize}
\renewcommand{\cftsubsectionfont}{\normalfont}
\renewcommand{\ABNTEXsubsubsectionfont}{\slshape}
\renewcommand{\cftsubsubsectionfont}{\normalfont\slshape}
\renewcommand{\ABNTEXsubsubsubsectionfont}{\bfseries}
\renewcommand{\ABNTEXsubsectionfont}{\bfseries}

% Para configurar mais níveis configure conforme utilizado acima e comente as duas linhas abaixo
\settocdepth{subsubsection} % configura sumário para apresentar subseções até o quarto nível
\setsecnumdepth{subsubsection} % configura para numerar subseções até o quarto nível. Subseções de quinto nível não conterão numeração.

\addto\captionsbrazil{\renewcommand{\listfigurename}{Lista de figuras}} % Altera nome da lista de ilustrações para lista de figuras

\addto{\captionsbrazil}{\renewcommand{\bibname}{Refer\^encias Bibliogr\'aficas}} % Altera nome Referencias para Referencias Bibliogarficas

% ----------------------------------------------
% Para criar Quadros - ABNT 14724 5.10
% ----------------------------------------------
\newcommand{\quadroname}{Quadro}
\newcommand{\listofquadrosname}{Lista de quadros}
\newfloat[chapter]{quadro}{loq}{\quadroname}
\newlistof{listofquadros}{loq}{\listofquadrosname}
\newlistentry{quadro}{loq}{0}
\counterwithout{quadro}{chapter}
\renewcommand{\cftquadroname}{\quadroname\space}
\renewcommand*{\cftquadroaftersnum}{\hfill--\hfill}

% ----------------------------------------------
% Para criar Gráficos
% ----------------------------------------------
\newcommand{\graficoname}{Gráfico}
\newcommand{\listofgraficosname}{Lista de gráficos}
\newfloat[chapter]{grafico}{logf}{\graficoname}
\newlistof{listofgraficos}{logf}{\listofgraficosname}
\newlistentry{grafico}{logf}{0}
\counterwithout{grafico}{chapter}
\renewcommand{\cftgraficoname}{\graficoname\space}
\renewcommand*{\cftgraficoaftersnum}{\hfill--\hfill}

% ----------------------------------------------
% Equações com numeração sequencial
% ----------------------------------------------
\counterwithout{equation}{chapter}

% ----------------------------------------------
% Pacotes básicos 
% ----------------------------------------------
\usepackage{lastpage}		% Usado pela ficha catalografica
\usepackage{indentfirst}	% Indenta o primeiro paragrafo de cada seção.
\usepackage{color}			% Controle das cores
\usepackage{graphicx}		% Inclusão de gráficos
\usepackage{microtype} 		% para melhorias de justificação
\usepackage{array}
%\usepackage{gensymb}       % Símbolos

\usepackage{amsmath} 	%--------------------------%
\usepackage{hyperref} 	%--------------------------%
\usepackage{bibentry} 	% para inserir refs. bib. no meio do texto


% ----------------------------------------------
% Pacotes adicionais
% ----------------------------------------------
\usepackage{lipsum}				% para geração de dummy text
\usepackage[colorinlistoftodos, english]{todonotes}
% uso: \todo[inline, color=red!80]{texto}
\usepackage{verbatim}
\usepackage{soulutf8}
% uso: \hl{highlight} ou \st{strikeout} ou \ul{underline}
%\usepackage{tabularx}
\usepackage{rotating,tabularx}
\usepackage{multirow}
\usepackage{multicol}
\usepackage{lscape}
\usepackage{varwidth}
\usepackage{subfig}
\usepackage{pdfpages}
\usepackage{pgfplots}
\pgfplotsset{compat=1.12}
\usepackage{float}
\usepackage[nohyperlinks]{acronym}
%\usepackage{subfigure}

% Para desenho de circuitos
\usepackage{tikz}
\usepackage[american]{circuitikz}
\usepackage{siunitx}
\usepackage{colortbl}

%para usar codeblock
\usepackage{listings}
\definecolor{dkgreen}{rgb}{0,0.6,0}
\definecolor{gray}{rgb}{0.5,0.5,0.5}
\definecolor{mauve}{rgb}{0.58,0,0.82}
\lstset{frame=tb,
  language=C,
  aboveskip=3mm,
  belowskip=3mm,
  showstringspaces=false,
  columns=flexible,
  basicstyle={\small\ttfamily},
  numbers=left,
  numberstyle=\tiny\color{gray},
  keywordstyle=\color{blue},
  commentstyle=\color{dkgreen},
  stringstyle=\color{mauve},
  breaklines=true,
  breakatwhitespace=true,
  tabsize=3
} 

\usepackage{booktabs}
\usepackage{adjustbox}
\usepackage{placeins}
\usepackage{longtable}
\usepackage{caption}
\usepackage{amssymb}

% ----------------------------------------------
% Pacotes de citações
% ----------------------------------------------
\usepackage[brazilian,hyperpageref]{backref}	 % Paginas com as citações na bibl
\usepackage[alf]{abntex2cite}	% Citações padrão ABNT

% ==============================================
% CONFIGURAÇÕES DE PACOTES
% ==============================================

\DeclareCaptionFormat{caption_with_break_line}{%
  % #1: label (e.g. "Table 1")
  % #2: separator (e.g. ": ")
  % #3: caption text
  \begin{varwidth}{\linewidth}%
    \centering
    #1#2#3%
  \end{varwidth}%
}

% ----------------------------------------------
% Configurações do pacote backref 
% usado sem a opção hyperpageref de backref
% ----------------------------------------------
\renewcommand{\backrefpagesname}{Citado na(s) página(s):~}
% Texto padrão antes do número das páginas
\renewcommand{\backref}{}
% Define os textos da citação
\renewcommand*{\backrefalt}[4]{
	\ifcase #1 %
		Nenhuma citação no texto.%
	\or
		Citado na página #2.%
	\else
		Citado #1 vezes nas páginas #2.%
	\fi}%

% ----------------------------------------------
% Espaçamentos entre linhas e parágrafos 
% ----------------------------------------------
% O tamanho do parágrafo é dado por:
\setlength{\parindent}{1.3cm}

% Controle do espaçamento entre um parágrafo e outro:
\setlength{\parskip}{0.2cm}  % tente também \onelineskip

% impedir quebra de notas de rodapé entre páginas
\interfootnotelinepenalty=10000

% ----------------------------------------------
% compila o indice
% ----------------------------------------------
\makeindex

% ||||||||||||||||||||||||||||||||||||||||||||||
% Informações de dados para CAPA e FOLHA DE ROSTO
% ||||||||||||||||||||||||||||||||||||||||||||||
\titulo{EDUCAÇÃO FINANCEIRA NO ENSINO MÉDIO INTEGRADO DOS INSTITUTOS FEDERAIS: UMA PROPOSTA ASSISTIDA POR JOGOS COMO TECNOLOGIA EDUCACIONAL} % Não utilize o ponto final no título
\autor{Johnata Souza Santicioli}
\local{Porto Alegre}
\data{2020}
\orientador{Prof. Dr. Evandro Manara Miletto}
\coorientador{Profa. Dra. Josiane Carolina S. R. Procasko} % comente esta linha caso nao tenha coorientador
\instituicao{%
  INSTITUTO FEDERAL DO RIO GRANDE DO SUL
  \par
  CAMPUS PORTO ALEGRE
  \par
  PROGRAMA DE MESTRADO PROFISSIONAL EM 
  \par
  INFORMÁTICA NA EDUCAÇÃO
  } 
\tipotrabalho{Dissertação (Mestrado)}
% O preambulo deve conter o tipo do trabalho, o objetivo, 
% o nome da instituição e a área de concentração 

%Preâmbulo da Qualificação
\preambulo{Proposta de Dissertação apresentada junto ao Programa de Pós-graduação Stricto Sensu – Mestrado Profissional em Informática na Educação do Instituto Federal de Educação, Ciência e Tecnologia do Rio Grande do Sul – Campus Porto Alegre, como requisito parcial ao desenvolvimento da Dissertação.}

%Preâmbulo da Dissertação
%\preambulo{Dissertação apresentada ao Programa de Pós-graduação Stricto Sensu – Mestrado Profissional em Informática na Educação do Instituto Federal de Educação, Ciência e Tecnologia do Rio Grande do Sul – Campus Porto Alegre, como parte dos requisitos para obtenção do título de Mestre em Informática na Educação.}

%\preambulo{Trabalho apresentado como requisito para a obtenção do título de Mestre, pelo Programa de Pós-Graduação em Engenharia Elétrica do Instituto Federal de Educação, Ciência e Tecnologia do Rio Grande do Sul – Campus Porto Alegre.}

% ----------------------------------------------
% Configurações de aparência do PDF final
% ----------------------------------------------
% alterando o aspecto da cor azul
\definecolor{blue}{RGB}{41,5,195}

% alterando o aspecto da cor cinza
\definecolor{gray}{RGB}{50,50,50}

% informações do PDF
\makeatletter
\hypersetup{
     	%pagebackref=true,
		pdftitle={\imprimirtitulo}, 
		pdfauthor={\imprimirautor},
    	pdfsubject={\imprimirpreambulo},
	    pdfcreator={LaTeX - abnTeX2 - Overleaf},
        pdfkeywords={abnt}{latex}{abntex2}{trabalho acadêmico}{ifsul}{mpie}{mestrado profissional}, 
		colorlinks=true, % false: boxed links; true: colored links
    	linkcolor=black, % color of internal links
    	citecolor=black, % color of links to bibliography
    	filecolor=blue,  % color of file links
		urlcolor=gray,	 % color of url links
		bookmarksdepth=4
}
\makeatother

\begin{document}
\input{2-pretextual.tex}
\textual
\chapter{Introdução}
Para parte da população brasileira a educação financeira e a matemática financeira são sinônimas, contudo, são ciências distintas que se complementam. A educação financeira trata do conhecimento sobre o comportamento, emoções e hábitos financeiros, enquanto a matemática financeira trata do conhecimento técnico, teórico e de fórmulas matemáticas relacionada à mudança de valor do dinheiro com a evolução do tempo \cite{puccini2011}. Segundo a OCDE - Organização para a Cooperação e Desenvolvimento Econômico (em inglês, OECD - \textit{Organisation for Economic Co-operation and Development}), a educação financeira é a melhoria do processo pelo qual os cidadãos compreendem sobre produtos, conceitos e riscos do mercado financeiro, obtendo informação, instrução, habilidades e confiança para realizar compras e investimentos de forma conscientes \cite{ocde2005}.

É notório que na estrutura social na qual vivemos a educação financeira é essencial para manter o bem-estar, equilíbrio e a manutenção de uma vida tranquila. Nesse sentido, saber lidar com dinheiro e realizar uma boa administração financeira influencia diretamente na qualidade de vida das pessoas, por isso, essa área de conhecimento tem ganhado grande importância dentro das famílias, principalmente devido ao aumento progressivo da complexidade dos produtos e mercados financeiros. A educação financeira também tem recebido incentivo de pessoas, governos, empresas e entidades em diversos países do mundo na disseminação de seus conteúdos.

No Brasil a implantação do Plano Real na década de 1990 promoveu uma estabilidade econômica no país, permitindo assim a elaboração de múltiplas iniciativas para promover a educação financeira propondo diversas reflexões sobre o tema. Essa iniciativa não era possível antes da criação do plano em consequência do consumo imediatista generalizado causado por uma hiperinflação que diminuía o poder de compra da população no mesmo dia do recebimento da renda \cite{souza2016}. Atualmente existem projetos de leis, estratégias, comitês que apoiam o desenvolvimento da educação financeira no país, corroborando com a proposta da Base Nacional Comum Curricular - BNCC que tornou o conteúdo obrigatório no ensino básico, entretanto são poucas as escolas que aplicam o ensino da educação financeira, inclusive em instituições reconhecidas pela sua qualidade como os da Rede Federal de Educação Profissional, Científica e Tecnológica (RFEPCT).

A RFEPCT é composta em sua maior parte pelos Institutos Federais de Educação, Ciência e Tecnologia uma autarquia federal de educação superior, básica e profissional, pluricurricular e multicampi, especializada na educação profissional e tecnológica nas modalidades de ensino médio técnico, superior e de pós-graduação, unindo em suas práticas pedagógicas os conhecimentos técnicos e tecnológicos. A rede oferece 50\% de suas vagas a cursos de ensino médio técnico integrados, analisando os projetos pedagógicos destes cursos observou-se o conteúdo sobre educação financeira, em partes, apenas nos poucos cursos intimamente ligados às finanças.

O processo de ensino-aprendizagem sobre educação financeira é importante para a formação de um profissional cidadão, principalmente no complexo sistema financeiro econômico no qual vivemos, restringir o acesso apenas a alguns alunos é podar o desenvolvimento e o saber dos demais, por isso a importância de ampliar o acesso a todos os alunos dos cursos de ensino médio integrado ao técnico da RFEPCT.

No propósito de encontrar uma metodologia com recursos e/ou métodos didáticos pedagógicos e dinâmicas competentes que possam auxiliar, motivar, apoiar, dirimir os problemas e dificuldades no processo de ensino-aprendizagem da Educação Financeira, busca-se nesse projeto, por meio de Jogos Sérios, um meio de desenvolver nos aluno do Ensino Médio Integrado da RFEPCT as competências e habilidade importantes para lidar com finanças pessoais.

\section{PROBLEMA DE PESQUISA}
Os alunos do ensino médio integrado da RFEPCT, em sua maior parte, possuem idades entre 14 a 19 anos, normalmente nesta fase eles estão no início de sua vida profissional e financeira, com pouca ou nenhuma experiência sobre educação financeira, como ensinar conteúdos tão complexos sobre comportamento e mercado financeiro a estes jovens e adolescentes?

O presente trabalho apresenta uma proposta metodológica assistida por tecnologias educacionais, como simuladores, ambientes virtuais e principalmente jogos para simplificar o conteúdo e facilitar a compreensão do discentes no processo aprendizagem sobre educação financeira.

Os autores deste projeto acreditam na hipótese de que uma metodologia assistida por Jogos Sérios e Simuladores, inovadoras ou não, possam dirimir a complexidade do mundo financeiro, tornando o conteúdo mais próximo dos estudantes, facilitando a compreensão, assimilação e acomodação de forma lúdica sobre temas fundamentais como finanças pessoais e mercado financeiro, demonstrando a importância e a influência deste conhecimento em suas vidas.

\section{OBJETIVOS}
Considerando o contexto atual descrito na introdução e no problema de pesquisa, este projeto possui objetivos primários e o secundários.

\subsection{Objetivo Primários}
Através desta pesquisa, almeja-se obter um método didático, assistido por Jogos Sérios, capaz de auxiliar no processo de aprendizagem da Educação Financeira para o Ensino Médio Integrado da RFEPCT. Com o intuito de fomentar nos alunos e seus familiares melhor compreensão sobre os conceitos e riscos do mercado financeiro, espera-se desta forma incentivá-los a obter instrução, adquirir habilidades, competências e confiança para organizar as finanças pessoais, bem como consumir e investir de forma consciente. Acredita-se que esse aporte de conhecimento contribuirá para a formação de um estudante profissional mais completo e preparado para enfrentar os desafios do mundo contemporâneo e globalizado. Faz parte também dos objetivos primários:
\begin{enumerate}
    \item [A.] Incentivar na disseminação dos conhecimentos sobre educação financeira;
    \item [B.] Promover a educação financeira dos discentes do Instituto Federal Sul-rio-grandense (IFSUL) Câmpus Gravataí;
    \item [C.] Proporcionar aos discentes conhecimentos para melhoria de hábitos no relacionamento com o dinheiro;
    \item [D.] Ensino e aprendizagem de gestão financeira pessoal;
    \item [E.] Ensino e aprendizagem iniciais sobre matemática financeira, mercado financeiro, investimentos.
\end{enumerate}

\subsection{Objetivos Secundários}
Com base na metodologia elaborada pelos autores do projeto e dos discentes que participarão da pesquisa, os objetivos secundários consistem em verificar:
\begin{enumerate}
    \item [A.] O nível aprendizagem dos participantes com a metodologia;
    \item [B.] Se as tecnologias educacionais utilizadas na metodologia de ensino facilitam o processo de aprendizagem;
    \item [C.] O interesse dos participantes na busca de novos conhecimentos após utilização da metodologia;
    \item [D.] A satisfação dos participantes com metodologia proposta;
    \item [E.] Se os participantes consideram o tema importante para suas vidas.
\end{enumerate}

\section{JUSTIFICATIVA}
As diversas iniciativas voltadas à promoção da educação financeira no Brasil, apontadas na Introdução, não são suficientes, basta analisar duas pesquisas de instituições reconhecidas: a primeira realizada pelo Serviço de Proteção ao Crédito (SPC) e a Confederação Nacional de Dirigentes Lojistas (CNDL) onde, “Em maio de 2018, o número de brasileiros negativados chegou a 63,29 milhões, o que representa 41,4\% da população adulta.” \cite{spc2018}. Ou seja,  mais de 40\% da população adulta brasileira não consegue arcar com suas dívidas e por consequência os seus nomes são cadastrados em listas de inadimplentes. Isto ocorre por diversos motivos, inclusive pela falta de conhecimento sobre educação financeira; a segunda pesquisa foi efetuada pela Organização para a Cooperação e Desenvolvimento Econômico (OCDE) na qual apurou-se o letramento financeiro de jovens de 15 anos dos países membros e parceiros. Esta pesquisa teve como finalidade  averiguar a capacidade dos jovens participantes em lidar com situações cotidianas que envolvam dinheiro, gastos, poupança e investimentos, habilidade extremamente importante ao considerar o complexo sistema financeiro brasileiro e mundial, o resultado da pesquisa aponta o Brasil em último lugar nas questões sobre o letramento financeiro, conforme Tabela \ref{tab: OECD, PISA 2015, Figure IV.1.1, p.31}.

\graphicspath{{tabelas/}}
\begin{table}[!ht]
\centering
\begin{minipage}{1.\textwidth}
\caption{Performance em letramento financeiro da OCDE}
\centering
\includegraphics[width=1.0\textwidth]{tabela01-pisa2015.jpg}
\legend{\footnotesize Fonte: \citeauthor{pisa2015}, PISA 2015, Figure IV.1.1,p.31}
\label{tab: OECD, PISA 2015, Figure IV.1.1, p.31}
\end{minipage}
\end{table}

As duas pesquisas demonstram a importância de fomentar a Educação Financeira na formação do cidadão brasileiro, para que este possa obter conhecimento sobre o complexo sistema financeiro que interfere diretamente em sua vida. Entretanto, nos currículos dos cursos no Ensino Médio Integrado da RFEPCT não consta a Educação Financeira como disciplina, exceto nos projetos pedagógicos  das instituições que possuem cursos na área de finanças como o Técnico em Administração ou Logística Integrados ao Ensino Médio.

Deste modo, projetos de pesquisa que visam incentivar, desenvolver e explorar metodologias no processo de ensino e aprendizagem voltados à Educação Financeira são extremamente importantes para facilitar a obtenção do conhecimento, para que o estudante possa relacionar-se melhor com o dinheiro e usufruir serviços financeiros de forma consciente, contribuindo deste modo também para a economia nacional.

\section{ESTRUTURA DO TRABALHO}
Este trabalho está organizado em seis capítulos, com o intuito de apresentar a bases teóricas e tecnológicas necessárias para a elaboração do Curso de Educação Financeira para Jovens nas instituições da RFEPCT.

O Capítulo 1 é composto pela a introdução, a hipótese, o problema de pesquisa, os objetivos do estudo (primários e secundários), a justificativa e a descrição da estrutura do texto deste trabalho.

Já o Capítulo 2 aborda a respeito do referencial teórico relativos aos temas fundamentais do estudo: Educação Financeira, as Políticas da Educação Financeira Nacional adotadas pelo governo brasileiro, a composição da RFEPCT e os Institutos Federais, a construção do saber da Educação Financeira através da Epistemologia Genética e Construtivismo de Piaget, por derradeiro as Tecnologias Educacionais utilizadas na concepção da metodologia por intermédio dos Jogos Sérios.

O Capítulo 3 contempla os aspectos metodológicos de tipologia da pesquisa quanto à sua natureza, aos objetivos, a análise dos dados, à finalidade e procedimentos.

No Capítulo 4 são apresentados os trabalhos relacionados com a Educação Financeira, Tecnologias Educacionais e o Ensino Médio Integrado nos Institutos Federais.

Em seguida, no Capítulo 5 são expostos os detalhes do Curso de Educação Financeira para Jovens, com as unidades temáticas, atividades e os conteúdos específicos. Por fim, o Capítulo 6 relata os resultados parciais da pesquisa.

\chapter{REFERENCIAL TEÓRICO}
No presente capítulo serão abordados os aspectos teóricos e tecnológicos inerentes ao tema da pesquisa, à elaboração da metodologia e ao Curso de Educação Financeira para Jovens da RFEPCT.

\section{EDUCAÇÃO FINANCEIRA}
Ao analisar algumas bibliografias e projetos acerca da educação financeira é possível elencar alguns assuntos que são tratados, os temas principais abordados são o Consumo Consciente (Desejos x Necessidades), Planejamento Orçamentário, Crédito e Dívidas, Poupança e Investimentos, Seguros e Previdência que serão descritos no decorrer deste capítulo, tais assuntos serão objetos de estudo no processo de ensino e aprendizagem do presente projeto.

\subsection{Consumo Consciente}
Segundo Buaes, Comerlato e Doll (\citeyear{buaes2015}) “entre o sonho de ter e a possibilidade de comprar é possível planejar o consumo. Esse é chamado de consumo consciente ou consumo planejado”. Em outras palavras, o consumo consciente, em finanças, compreende a utilização dos recursos de forma planejada, com planos e prazos bem definidos para evitar gastos desnecessários ou desperdício de recursos.

Contudo, diversas estratégias de vendas são utilizadas para estimular o consumidor a realizar gastos não planejados, tais estímulos são chamados de Gatilhos Mentais. O psicólogo Kahneman (\citeyear{kahneman2012}) sugere que o ser humano possui duas formas distintas para planejar e executar as ações: o Pensamento Rápido e o Pensamento Lento. O primeiro é emocional, intuitivo, impulsivo, involuntário e valoriza o presente, enquanto o segundo é racional, dedutivo, lento, consciente e valoriza o futuro.

Estratégias de vendas utilizam Gatilhos Mentais para estimular a compra, visando se aproveitar do Pensamento Rápido. Segundo Cialdini (\citeyear{cialdini2012}), há diversos gatilhos mentais, por exemplo o Gatilho da Escassez que ocorre quando um produto ou serviço é divulgado com poucas unidades, ou o Gatilho da Urgência quando a oportunidade de compra é limitada por um pequeno interstício. Ambos gatilhos mentais  exploram a aversão à perda do ser humano, desta forma os clientes são estimulados a adquirir produtos ou serviços, por fim efetuam a compra baseado no Pensamento Rápido que foi incentivado por um Gatilho Mental, sem o cliente perceber, por isso a importância de planejar o consumo, através de um Planejamento Orçamentário.

\subsection{Planejamento Orçamentário}
O planejamento orçamentário ou orçamento é uma ferramenta para o planejamento financeiro pessoal ou familiar onde deve constar todas as receitas fixas e variáveis (entrada monetária), despesas fixas e variáveis (saída monetária) utilizado para decidir quais serão as prioridades de consumo pessoal ou de uma família.

\begin{citacao}
O orçamento permite o planejamento de como gastar o seu dinheiro e mesmo economizar e investir. Após listar detalhadamente todas as receitas e despesas, é preciso fazer o balanço do mês, para saber quanto sobra, quanto falta ou se há equilíbrio entre ganhos e gastos. O orçamento possibilita o planejamento financeiro, ou seja, escolher em que e como vai gastar a partir da definição de suas prioridades, além de ajudar a administrar os imprevistos e reduzir o consumo desnecessário e indesejado. \cite{buaes2015}
\end{citacao} 

Por isso é tão importante o planejamento orçamentário na realização do consumo consciente, sonhos, controle financeiro pessoal e familiar, sem o orçamento é maior a possibilidade endividamento excessivo e posteriormente a inclusão do nome em lista de proteção de crédito.

O orçamento pode ser dividido em 4 (quatro) etapas: planejar, registrar, agrupar e avaliar. Planejar consiste no levantar todas as receitas e despesas, estabelecendo um limite para cada categoria de gasto, o ideal é que as despesas não ultrapassem as receitas e quanto menor for melhor, pois assim sobrará mais verba para poupar. Registrar todos as despesas e receitas diárias inclusive os valores menores que somados geram gastos consideráveis e por muitas vezes não são observados, esse registro pode ser feito através de planilhas eletrônicas, aplicativos mobile ou qualquer software para facilitar as anotações. Agrupar as despesas e receitas anotadas em grupo, ou seja, separar cada movimentação em categorias como Alimentação, Moradia, Saúde, Mercado, Compras, Educação, Rendimento, Remuneração etc., a organização de categorias pode também ser classificada no momento do registro. Por último avaliar se as metas máximas de consumo e receitas planejadas foram alcançadas em todas categorias, o ideal é que a soma dos gastos não ultrapasse o valor das receitas, a repetição contínua deste ciclo provê a continuação de um consumo consciente evitando a solicitação de créditos o que pode causar dívidas.

\subsection{Crédito e Dívidas}
A utilização do crédito é muito comum entre a população brasileira para realizar compras de produtos, serviços, bens de consumo, imóveis, veículos etc., contudo o que é o crédito? Segundo o Banco Central do Brasil (\citeyear{bacen2013} p. 25): “O crédito é uma fonte adicional de recursos que não são seus, mas obtidos de terceiros (bancos, financeiras, cooperativas de crédito e outros), que possibilita a antecipação do consumo para aquisição de bens ou contratação de serviços”, em outros termos, crédito é o empréstimo de valor monetário de terceiros a consumidores quando estes não possuem ou não querem utilizar recursos próprios para aquisição de bens, serviços, produtos e outros. Existem diversas modalidades de crédito como cheque especial, empréstimos pessoais, financiamentos, cartão de crédito, crédito consignado, compra a prazo em lojas, entretanto toda a modalidade de adiantamento financeiro cobra um valor adicional além do emprestado, chamado de juros, quantia cobrada pelo aluguel do dinheiro ao consumidor que encarece o preço final.

A dívida é todo o compromisso financeiro realizado através de crédito em compras parceladas de produtos, serviços, bens de consumo, imóveis etc., o excesso de dívidas, a falta de consumo consciente e de um plano orçamentário pode acarretar  no atraso de pagamentos, ou pior, o não pagamento das dívidas e neste momento o consumidor tornar-se-á inadimplente passível de ter o seu nome inscrito em órgãos de proteção ao crédito como o SCPC (Serviço Central de Proteção ao Crédito) que dificultará a obtenção de novos créditos, a melhor forma de comprar um produto ou serviço é poupar e investir mensalmente até obter o valor total do item, evitando contratar créditos que comprometem o poder de compra dos meses posteriores.

\subsection{Poupança e Investimentos}
O orçamento pessoal ou familiar pode ser avaliado de forma geral calculando a diferença entre o total de entrada monetária (receita) e o total de saída monetária (despesa), desta forma quando:
\begin{itemize}
    \item A despesa é maior que a receita o orçamento é deficitário;
    \item A despesa for igual a receita o orçamento é neutro;
    \item A receita é maior que a despesa o orçamento é superavitário.
\end{itemize}

Segundo o Banco do Brasil (\citeyear{bb2012}, p. 24): “Quando a pessoa não consome toda a sua renda, essa sobra poderá ser transformada em poupança”, portanto todo orçamento superavitário produz uma sobra financeira chamada de poupança, normalmente confundido com Caderneta de Poupança que é o investimento mais popular no Brasil.

O investimento ou aplicação ocorre quando o dinheiro poupado é utilizado para gerar mais dinheiro, por isso é considerado um ativo, existem diversos tipos de aplicações financeiras no Brasil os mais comuns são a Caderneta de Poupança, Certificados de Depósito Bancário (CDB), Recibo de Depósito Bancário (RDB), Letras de Crédito (LC), Letras de Crédito Imobiliário (LCI) e Letras de Crédito do Agronegócio (LCA), Debêntures, Títulos Públicos, Ações, Fundos e Clubes de Investimentos.

Todo o investimento possui três componentes liquidez, risco e rentabilidade, de acordo com Banco Central do Brasil - BACEN (\citeyear{bacen2013}) a liquidez é a capacidade de um investimento ser transformado em dinheiro, em qualquer tempo por um preço justo, o risco é a possibilidade de ocorrer perda no valor investido e a rentabilidade é a remuneração que o investimento retorna ao investidor. Quanto maior o risco, maior é a possibilidade de ocorrer perdas de capital, entretanto maior será a rentabilidade também.

Os investimentos podem ser de renda fixa e/ou renda variável, na renda fixa a remuneração paga é correspondente a uma taxa de juros determinada na contratação, já na renda variável não é possível determinar uma taxa de remuneração na contração \cite{bacen2013}. Em outras palavras o investidor que contrata uma aplicação de renda fixa sabe qual será a taxa de juros acordada, ao contrário da renda variável onde a rentabilidade será determinada no momento da liquidez, o que pode inclusive incorrer em perdas do montante aplicado. Ao realizar um plano de investimento é importante considerar o risco, a liquidez, a rentabilidade, o tipo da aplicação e a finalidade dos recursos investido de acordo com o objetivo de cada investidor.

\subsection{Seguros e Previdência}
Segundo o site da Superintendência de Seguros Privados (SUSEP), órgão federal do Brasil responsável pelo controle e fiscalização das seguradoras, empresas que oferecem este tipo de contrato, seguro é o:
\begin{citacao}
Contrato mediante o qual uma pessoa denominada Segurador, se obriga, mediante o recebimento de um prêmio, a indenizar outra pessoa, denominada Segurado, do prejuízo resultante de riscos futuros, previstos no contrato \cite{susep2007}.
\end{citacao}

A vida é imprevisível, em boa parte, eventos não planejados, inesperados ou impensáveis podem ocorrer com qualquer ser humano, imprevistos positivos ou negativos, entre os eventos ruins pode-se citar como exemplo, doenças, acidentes, morte, desemprego, furtos ou roubos de bens entre outros.

Por isso a importância da contratação de um seguro, com objetivo de diminuir o prejuízo financeiro causado por eventuais problemas. As seguradoras oferecem diversos tipos de seguros com respectivos planos, benefícios e contratos, entre os principais destaca-se o de vida, residencial, veicular, empresarial e viagem.

Da mesma forma a previdência em finanças é a preparação econômica durante um determinado período para lidar com situações do futuro como aposentadoria ou realização de projetos \cite{bb2015}, atualmente existem dois tipos de regimes para prover recursos financeiros a Previdência Social, administrada pelo Instituto Nacional de Seguro Social, e a Previdência Privada, oferecida por bancos e seguradoras e supervisionada pela SUSEP.

\section{POLÍTICAS DA EDUCAÇÃO FINANCEIRA NACIONAL}
O intuito deste capítulo é abordar as políticas públicas educacionais do Brasil voltadas para o desenvolvimento da educação financeira e descrever os projetos propostos para todo o território, especificamente os projetos de leis sobre a educação financeira que tramitam no Congresso Nacional, a Estratégia Nacional de Educação Financeira - ENEF, o Comitê Nacional de Educação Financeira - CONEF, a Associação de Educação Financeira - AEF-Brasil, as propostas da Base Nacional Comum Curricular - BNCC para o ensino de finanças na educação básica e ao fim uma consideração sobre o desenvolvimento da educação financeira no país.

\subsection{Projetos de Lei Sobre a Educação Financeira}
Na Câmara dos Deputados, casa iniciadora de leis no Brasil, existem 12 projetos de lei cujo conteúdo propõem a inclusão ou criação da Educação Financeira como componente curricular na educação básica (Ensino Fundamental e o Ensino Médio), conforme o Quadro \ref{quad: projetos-lei}.

\begin{quadro}[!ht]
\caption{Projetos de Lei Sobre Educação Financeira na Câmara dos Deputados}
\centering
\begin{tabular}{|l|c|l|l|}\hline
\rowcolor[rgb]{0.867,0.922,0.969} \multicolumn{1}{|c|}{\textbf{Projeto de Lei}} & \textbf{Ano} & \multicolumn{1}{c|}{\textbf{Situação}} & \multicolumn{1}{c|}{\textbf{Fonte}}  \\
\hline \hline
PL 3401/2004 & 2004 & Arquivada & \cite{projeto3401} \\ \hline
PL 306/2007 & 2007 & Arquivada & \cite{projeto306} \\ \hline
PL 2107/2011 & 2011 & Em Pauta na Comissão de Educação (CE) & \cite{projeto2107} \\ \hline
PL 3421/2012 & 2012 & Apensado ao PL 2107/2011 & \cite{projeto3421} \\ \hline
PL 7155/2014 & 2014 & Apensado ao PL 2107/2011 & \cite{projeto7155} \\ \hline
PL 3590/2015 & 2015 & Apensado ao PL 2107/2011 & \cite{projeto3590} \\ \hline
PL 3691/2015 & 2015 & Apensado ao PL 3590/2015 & \cite{projeto3691} \\ \hline
PL 4215/2015 & 2015 & Apensado ao PL 2107/2011 & \cite{projeto4215} \\ \hline
PL 4915/2016 & 2016 & Apensado ao PL 2107/2011 & \cite{projeto4915} \\ \hline
PL 7318/2017 & 2017 & Apensado ao PL 2107/2011 & \cite{projeto7318} \\ \hline
PL 239/2019 & 2019 & Apensado ao PL 2107/2011 & \cite{projeto239} \\ \hline
PL 3114/2019  & 2019 & Apensado ao PL 7318/2017 & \cite{projeto3114} \\ \hline
\end{tabular}
\label{quad: projetos-lei}
\vspace{0.2cm} \legend{\footnotesize Fonte: O Autor, 2019}
\end{quadro}

Na construção do Quadro \ref{quad: projetos-lei} considerou-se apenas os projetos da Câmara dos Deputados, por ser a casa iniciadora dos projetos de leis federais. É possível observar que a educação financeira é tratada no Congresso Nacional desde 2004, entretanto, após 15 anos do primeiro projeto nenhum foi transformado em lei e outros dois foram arquivados, o único projeto que continua em pauta é o PL 2107/2011 que encontra-se na Comissão de Educação, o restante foram apensados em outros projetos.

O Projeto de Lei PL 2107/2011 que possui como proposta alterar a lei de diretrizes e bases da educação nacional, incluindo a disciplina de Noções de Educação Financeira como conteúdo obrigatório no Ensino Médio, além disso, estabelecer o interstício de 3 anos para os sistemas de ensino se adequarem às exigências descritas \cite{projeto2107}, todavia, não aborda a Educação Financeira para o Ensino Fundamental. É importante perceber através do Quadro \ref{quad: projetos-lei} que existe apenso ao PL 2107/2011 outros 7 projetos de Leis tornando-o principal projeto desta matéria no Congresso Nacional.

Ao analisar o histórico dos projetos de lei voltados à Educação Financeira, observa-se que primeiro foi proposto em 2004, há 15 anos, e até a elaboração deste trabalho nenhum projeto foi transformado em lei, fato que demonstra a falta de responsabilidade dos congressistas brasileiros com o desenvolvimento Educação Financeira no país.

\subsection{A ENEF}
Entre as diversas iniciativas para promover o letramento financeiro destaca-se a Estratégia Nacional de Educação Financeira (ENEF) instituída pelo governo brasileiro através Decreto nº7.397, de 22 de dezembro de 2010, sob a responsabilidade do Comitê Nacional de Educação Financeira (CONEF) com execução e coordenação dos programas realizadas pela Associação de Educação Financeira do Brasil (AEF-Brasil), tal estratégia tem a finalidade de fortalecer a cidadania, aumentar a eficiência e solidez do sistema financeiro, disseminar a educação financeira e previdenciária, promover as decisões financeiras de forma consciente e autônomas, bem como atuar com informação, orientação e formação de forma gratuita visando o interesse público, por intermédio de dois documentos norteadores: Orientações para Educação Financeira nas Escolas e Orientações para Educação Financeira de Adultos \cite{enef2010}. Através da ENEF são executados projetos de cursos, blogs, vídeos, livros, jogos, material de apoio aos professores do ensino médio e fundamental, a Semana da Educação Financeira e o Mapeamento de Iniciativas de Educação Financeira voltadas para o desenvolvimento da educação financeira.

O CONEF, responsável pela a ENEF, foi instituído pelo mesmo decreto que criou a ENEF sendo composto pelo Diretor do Banco do Brasil, Presidente da Comissão de Valores Mobiliários, Diretor-Superintendente da Superintendência Nacional de Previdência Complementar, Superintendente da Superintendência de Seguros Privados, Secretário-Executivo do Ministério da Fazenda, Secretário-Executivo do Ministério da Educação, Secretário-Executivo do Ministério do Trabalho e Previdência Social e o Secretário Nacional do Consumidor do Ministério da Justiça, observa-se nesta composição a presença dos principais órgãos federais relacionados às finanças, economia e trabalho. Compete ao comitê promover a educação financeira através de planos, metas, programas, ações, financiamento, execução de projetos e revisão da ENEF \cite{conef2010}.

A AEF-Brasil é uma Organização da Sociedade Civil de Interesse Público (OSCIP), com certificação do Ministério da Justiça, composta por quatro instituições do mercado financeiro: Associação Brasileira das Entidades dos Mercados Financeiros e de Capitais (ANBIMA); Brasil, Bolsa, Balcão (B3); Confederação Nacional das Empresas de Seguros Gerais, Previdência Privada e Vida, Saúde Suplementar e Capitalização (CNSeg); e Federação Brasileira de Bancos (FEBRABAN). Por convênio de cooperação técnica com o CONEF a AEF-Brasil é responsável por executar e coordenar todos os programas propostos pela ENEF e tem o objetivo de promover a Educação Financeira nacional e de tecnologias socioeducacionais nesta área \cite{aefbrasil2011}. A AEF-Brasil é uma  associação sem fim lucrativo e recebe apoio financeiro de instituições privadas, do governo e da sociedade civil que compreendem a importância da Educação Financeira.

Entregar parte da responsabilidade social e administrativa tem-se tornado comum em alguns países como o Brasil, principalmente em governos neoliberalistas, entretanto a entrega da execução e coordenação dos programas da ENEF à AEF-Brasil, causa um grande confronto de paradigmas, pois como poderá uma OSCIP, composta por empresas privadas que sobrevivem de lucros, responder por políticas públicas voltadas ao bem comum, coletivo e sem fins lucrativos? Principalmente em uma área primordial para o desenvolvimento de um país como é a Educação Financeira.

\subsection{A BNCC e a Educação Financeira}
A Constituição do Brasil de 1988 prevê a elaboração de conteúdos mínimos para o Ensino Fundamental a fim de assegurar formação básica comum nacional, entretanto somente em 1996 é aprovada a Lei 9.394/1996 que regula a base nacional comum para o ensino básico em todos sistemas de ensino \cite{brasil1996}, entre os anos 1997 à 2012 são lançados os Parâmetros Curriculares Nacionais (PCN) e as Diretrizes Curriculares Nacionais Gerais para o ensino básico (DCNs) e os Parâmetros Curriculares Nacionais para o Ensino Médio (PCNEM) “com o objetivo de cumprir o duplo papel de difundir os princípios da reforma curricular e orientar o professor, na busca de novas abordagens e metodologias.” \cite{brasil2017b}. Em 2014 o Ministério da Educação e Cultura iniciou a elaboração da BNCC visando estabelecer os ensinamentos básicos para o desenvolvimento de conhecimentos necessários para o exercício da cidadania. No ano de 2015 fora disponibilizado a primeira versão da base e consequentemente iniciou-se uma mobilização entre as instituições de ensino para a discussão sobre a BNCC. Após dois anos de debates e mudanças a base foi homologada pelo MEC em 20 de dezembro de 2017 e apresentava orientações sobre conceitos básicos da Educação Financeira como economia, finanças, taxas de juros, inflação, aplicações financeiras e impostos, com o intuito de unificar o processo de ensino e aprendizagem, além de tornar o ensino da Educação Financeira obrigatório em todas as instituições de ensino até 2020.

Segundo a BNCC nas séries iniciais do Ensino Fundamental, do 1ª ao 4ª ano, deve-se trabalhar com os estudantes especificamente a unidade temática “Grandezas e Medidas” concentrando-se no reconhecimento, conversões e resolução de problemas do sistema monetário brasileiro, habilidades estas que estão ligadas à Matemática Financeira.

E volta a partir do 6ºano e permanece até o fim do Ensino Fundamental através da unidades temáticas “Números” e “Grandezas e Medidas”, trabalhando as habilidades EF06MA13, EF06MA32, EF07MA02, EF07MA02 e a EF09MA05 que descreve:

\begin{citacao}
Resolver e elaborar problemas que envolvam porcentagens, com a ideia de aplicação de percentuais sucessivos e a determinação das taxas percentuais, preferencialmente com o uso de tecnologias digitais, no contexto da educação financeira. \cite{brasil2017c}.
\end{citacao}

A habilidade EF09MA05, é indicada na BNCC ao 9º do Ensino Fundamental e aborda parte da Educação Financeira através de uma metodologia ativa de resolução de problemas com apoio da tecnologia o que poderá trazer estímulos maiores para aprendizagem, principalmente se o contexto do problema proposto fizer parte do cotidiano do aluno.

No mesmo sentido, no Ensino Médio a BNCC apresenta a Matemática Financeira e parte da Educação Financeira por meio de temas ligados a finanças pessoais e familiares como orçamento, investimentos, sustentabilidade e tomada de decisões, também com apoio de tecnologias digitais através das habilidades descritas no Quadro \ref{quad: quadro-2_habilidadesbncc-medio}.

\graphicspath{{quadros/}}
\begin{quadro}[!ht]
\centering
\begin{minipage}{1.\textwidth}
\caption{Habilidades propostas para Matemática Financeira e Educação Financeira}
\centering
\includegraphics[width=0.8\textwidth]{quadro-2_habilidades.png}
\legend{\footnotesize Fonte: \cite{brasil2018} Adaptado pelo Autor}
\label{quad: quadro-2_habilidadesbncc-medio}
\end{minipage}
\end{quadro}

Analisando o Quadro \ref{quad: quadro-2_habilidadesbncc-medio} nota-se que as habilidades descritas e propostas na BNCC, em sua maior parte, são voltadas ao contexto da Matemática Financeira, exceto as habilidades EM13MAT101, EM13MAT104 e EM13MAT203 que abordam parcialmente a Educação Financeira.

Apesar das polêmicas e debates na construção da Base Nacional Comum Curricular, é possível depreender que esta apresenta pequenos avanços na construção da Educação Financeira Nacional, entretanto, é observado em sua construção vieses do mercado trabalho e empreendedorismo na concepção de suas habilidade e competência, todavia a Educação Financeira crítica que proporciona ao indivíduo melhor qualidade de vida e bem estar social é a que fará exercer melhores escolhas financeiras e fortalecer a cidadania do país.

\section{A RFEPCT E OS INSTITUTOS FEDERAIS}
A história da RFEPCT teve início em 1909 quando o Presidente Nilo Peçanha cria formalmente uma rede com 19 “Escolas de Aprendizes e Artífices”, uma em cada capital dos Estados da República, para o ensino profissional primário e gratuito, através do Decreto 7.566 em 23 de setembro \cite{brasil1909} as quais foram inauguradas em 1910, alcançando todas as regiões do Brasil.

A educação profissional passou por diversas mudanças das quais podem-se destacar: o acesso a todos os ramos e graus (Lei 378/1937); a autonomia na organização didática e gestão em 1959; a equiparação ao ensino acadêmico (Lei 4.024/1961); o currículo do segundo grau passa a ser todo técnico-profissional (Lei 5692/71); a instituição do Sistema Nacional de Educação Tecnológica (Lei 8.948/1994); a permissão do ensino médio integrado ao técnico (Decreto 5.154/2004);  a integração a Educação de Jovens e Adultos em 2006; a instituição legal da Rede Federal de Educação Profissional, Científica e Tecnológica - RFEPCT (Lei 11.892/2008). As “Escolas de Aprendizes e Artífices” também sofreram alterações em sua nomenclatura, a partir de 1937 denominou-as como “Liceus Profissionais” (Lei 378/1937), em 1942 “Escolas Industriais e Técnicas” (Decreto 4.127/1942), em 1959 “Escolas Técnicas”, a partir de 1978 são chamadas de Centros Federais de Educação Tecnológica - CEFETS, e por derradeiro, em 2008 Institutos Federais de Educação, Ciência e Tecnologia.

Com a promulgação da Lei 11.892 em 2008, a Rede Federal de Educação Profissional, Científica e Tecnológica foi formalmente oficializada, formada por um conjunto de autarquias da administração pública indireta que possuem autonomia administrativa, patrimonial, financeira, didático-pedagógica e disciplinar, composta pela seguintes instituições \cite{brasil2008}:

\begin{itemize}
    \item 38 Institutos Federais de Educação, Ciência e Tecnologia
    \item 3 Centros Federais de Educação Tecnológica
    \item 23 Escolas, Colégios, Centros vinculados às Universidades Federais
    \item Universidade Tecnológica Federal do Paraná
    \item Colégio Pedro II
\end{itemize}

\newpage

Os Institutos Federais tem por objetivos: ministrar educação profissional técnica de nível médio no formato integrado, subsequente e concomitante,  privilegiando os cursos integrados, reservando no mínimo 50\% para este objetivo; realizar projetos de extensão e pesquisas aplicadas; ministrar em nível de educação superior cursos de bacharelado, tecnologia e licenciatura, bem como pós-graduação \textit{lato sensu e stricto sensu} de mestrado e doutorado. Os objetivos devem estar em articulação com o mundo do trabalho, a sociedade e o desenvolvimento socioeconômico local e regional nas áreas da educação profissional e tecnológica \cite{brasil2008}. A tabela \ref{tab: tabela02-institutos} apresenta a disposição dos Institutos Federais e o Colégio Pedro II pelo Brasil.

\graphicspath{{tabelas/}}
\begin{table}[!ht]
\centering
\begin{minipage}{1.\textwidth}
\caption{Disposição dos Institutos Federais}
\centering
\includegraphics[width=0.7\textwidth]{tabela02-institutos.png}
\legend{\footnotesize Fonte: Do Autor}
\label{tab: tabela02-institutos}
\end{minipage}
\end{table}

Dentro dos diversos institutos há o Instituto Federal Sul-rio-grandense-IFSUL local onde será aplicado à pesquisa do presente trabalho, especificamente no Campus Gravataí.

\subsection{O IFSUL E O CAMPUS GRAVATAÍ}
O Rio Grande do Sul foi o último estado a possuir uma “Escolas de Aprendizes e Artífices” diferente dos outros estados do Brasil a escola iniciou suas aulas apenas no início em 1930, assim como outras escolas sofreu alterações políticas, cursos, público e em sua nomenclatura até 2008 quando tornou-se o Instituto Federal de Educação, Ciência e Tecnologia Sul-rio-grandense - IFSUL \cite{ifsul2015}. Atualmente o instituto possui cursos técnicos (integrados, concomitante e subsequente), graduação (licenciatura, bacharelado e tecnologia), pós-graduação \textit{lato sensu e stricto sensu} (mestrado), em diversas áreas do conhecimento, distribuídos em 14 \textit{campi} (Bagé, Camaquã, Charqueadas, Gravataí, Avançado Jaguarão, Lajeado, Avançado Novo Hamburgo, Passo Fundo, Santana do Livramento, Sapiranga, Sapucaia do Sul, Venâncio Aires e Pelotas).

O Campus Gravataí iniciou as atividades no segundo semestre de 2014, atualmente o campus possui dois cursos, Superior em Análise e Desenvolvimento de Sistemas (Tecnólogo) e Técnico de Informática para Internet (Integrado ao Médio) no qual será aplicada a pesquisa do presente trabalho.

\section{O CONSTRUTIVISMO DE PIAGET  E A EDUCAÇÃO FINANCEIRA}
Ao buscar informações sobre educação financeira são encontradas inúmeras fontes de livros, sites, vídeos, blogs, cursos etc. sobre planejamento financeiro orçamentário, sonhos, objetivos, metas, planos, aposentadoria, despesas, receitas, dívidas, gastos, poupança, investimentos, hábitos e psicologia financeira comportamental. Consumir essas informações podem exigir do sujeito um saber antecipado para o desenvolvimento de sua aprendizagem, o que pode dificultar a formação de estruturas cognitivas básicas necessárias para transformação da informação em conhecimento prático \cite{piaget2011}. Tendo em vista a importância de se estabelecer um caminho para a aprendizagem, este capítulo apresenta uma via de acesso para a construção do conhecimento sobre Educação Financeira, correlacionando seus níveis de conhecimento e aprendizagem com os estágios do desenvolvimento cognitivo.

Piaget (\citeyear{piaget1971}), propõem que o desenvolvimento cognitivo de uma criança possui 4 níveis ou estágios (sensório motor, pré-operatório, operatório concreto e operatório formal) com características específicas em cada uma das fases, como a idade dos indivíduos. Segundo a teoria piagetiana, o sujeito atinge o estágio operacional formal a partir dos 12 anos, entretanto, quando trata-se de um tema desconhecido e distante do sujeito, observam-se ações, atitudes e interações semelhantes aos primeiros estágios do desenvolvimento cognitivo, mesmo em sujeitos que por sua idade e maturação atingiram o último estágio (Operacional Formal), e assim quando necessita adquirir o conhecimento de um novo tema transitam dos estágios iniciais até o final, por meio de diversos processos de assimilação, acomodação, adaptação e equilíbrio para adquirir o novo saber, bem como colocá-lo em prática. De forma similar ocorre no processo de ensino e aprendizagem da Educação Financeira, independente da idade do sujeito.

O primeiro estágio proposto por Piaget (\citeyear{piaget1971}) é o sensório motor. Nessa fase a criança começa a construir esquemas para assimilar o meio através de percepções sensoriais e esquemas motores, é a inteligência prática do aqui e agora. Nesta etapa a criança não apresenta nenhuma conduta em relação aos objetos que desaparecem do seu campo visual, por isso não compreendem nem sabem para onde o objeto foi, normalmente continuam procurando-o no local onde o objeto desapareceu e não compreendem que este continua existindo em outro local, ocorrem também reações circulares secundárias onde qualquer ação que leva ao um fim agradável é repetida intencionalmente. 

De modo análogo no desenvolvimento da educação financeira no primeiro estágio o sujeito troca seu dinheiro baseado em sentimentos agradáveis que a compra lhe proporciona, em diversas ocasiões as sensações são iniciadas por estratégias de vendas baseadas em gatilhos (atalhos) mentais, o sujeito repete intencionalmente a ação e não percebe que o dinheiro diminui a cada gasto, sem saber também o destino do dinheiro quando questionado, pois não há administração, o salário termina e ele não tem ideia com que gastou. Define-se esse estágio como o \textbf{Consumidor Sensorial}, que normalmente lidam com o dinheiro através da inteligência prática do aqui e agora utilizam todo o dinheiro que ganha mensalmente, muitas vezes recorrendo à empréstimos se necessário para suprir seus desejos, para os sujeitos nesse estágio dinheiro é apenas uma objeto de troca para obter sensações agradáveis. Não poupam para uma emergência ou oportunidade, gastam sem estabelecer um plano orçamentário, não registram os gastos e receitas mensais, por isso tem dificuldade em observar e saber o destino do dinheiro utilizado.

O segundo estágio proposto por Piaget (\citeyear{piaget1971}) é o pré-operatório. Nessa fase a criança desenvolve o domínio da linguagem e da representação, possui característica da função simbólica (consegue gerar imagem mental e representar), jogo simbólico (brincadeira de faz de conta), desenhos e o pensamento intuitivo (conhecimento baseado na percepção, ainda não compreende a irreversibilidade) entre outros. Da mesma forma no desenvolvimento da educação financeira no segundo estágio o sujeito começa a desenvolver o domínio sobre o dinheiro e sensações que impulsionam seus gastos, consegue compreender que o dinheiro é um símbolo e na verdade representa o tempo de vida que se perde para obtê-lo, portanto entende que um produto ou serviço é pago com tempo de vida e não com dinheiro, geralmente por causa do desequilíbrio financeiro do estágio anterior começa a realizar anotações de despesas, receitas, desenhar um orçamento e estudar sobre educação financeira para poder assimilar, acomodar e adaptar os novos conhecimentos e assim realizar o equilíbrio do conhecimento financeiro, aprende também a viver de acordo com seu nível financeiro em outras palavras gastar somente aquilo que ganha, neste estágio o sujeito já pensa e representa seus sonhos e objetivos em relação ao dinheiro, todavia não consegue elaborar metas e realizá-los pois não compreende ainda a irreversibilidade de suas atitudes financeiras por isso ainda não aplicam ou investem para conseguir alcançar seus sonhos e objetivos. Define-se esse estágio como o \textbf{Consumidor Pré-Poupador}. Normalmente não poupam dinheiro para objetivos como a aposentadoria, honram os compromissos financeiros com facilidade, não aplicam nem investem utilizam de todos os recursos financeiros adquiridos mensalmente.

O terceiro estágio proposto por Piaget (\citeyear{piaget1971}) é o operatório concreto nessa fase o pensamento é lógico, objetivo e reversível, a criança não se limita a uma representação imediata mais ainda depende do mundo concreto para chegar à abstração. De igual modo no terceiro estágio da educação financeira o sujeito desenvolve o pensamento lógico, objetivo e reversível, através desse pensamento compreende a necessidade de poupar dinheiro para o sonhos, desejos, imprevistos e aposentadoria bem como a importância de adquirir serviços de seguros, entretanto ainda não consegue elaborar projetos, metas e prazos para alcançar seus sonhos e objetivos, define-se este estágio como o \textbf{Poupador Concreto}. Nesse estágio o sujeito realiza o controle efetivo das receitas e despesas pessoais e familiar, vive no mínimo um nível financeiro abaixo da sua renda em outras palavras suas despesas mensais são sempre menores que as receitas obtendo um orçamento superavitário, poupando dinheiro na maior parte dos meses, normalmente aplicam o dinheiro que sobrou em uma caderneta de poupança, por falta de conhecimento em outros investimentos e pelo contexto histórico financeiro brasileiro, onde a população tem preferência por este tipo de investimento. Aplicam o dinheiro poupado na caderneta de poupança e normalmente não definem metas de curto, médio e longo prazo.

O quarto estágio proposto por Piaget (\citeyear{piaget1971}) é o operatório formal. Nessa fase a criança desenvolve-se o pensamento hipotético-dedutivo, bem como a consolidação da personalidade, socialização, autonomia, moralidade, liberdade e direitos. Igualmente no quarto estágio da educação financeira o sujeito desenvolve o pensamento hipotético-dedutivo desta forma consegue abstrair seus sonhos e objetivos financeiros criando metas de curto, médio e longo prazo para alcançá-los com o dinheiro poupado mensalmente, também consolida a personalidade de investidor (conservador, moderado ou arrojado), socialização para aprender e ensinar mais sobre finanças, autonomia e liberdade financeiras. Define-se esse estágio como o \textbf{Investidor Formal}. Nesse estágio o sujeito estabelece plano com prazos e metas bem definidos para alcançar seus objetivos financeiros, reconhece seu perfil de investidor e consome os melhores serviços financeiros de acordo com seu perfil e objetivos, busca constantemente obter novos conhecimentos sobre educação financeira, entende e utiliza seus direitos. Normalmente possuem o montante investido para reserva de emergência ou oportunidade, poupam parte do dinheiro recebido mensalmente para sonhos e objetivos, registram objetivos e metas de curto, médio e longo prazo, não possuem dívidas, reconhecem seu perfil de investidor, aplicam o dinheiro poupado em diversos investimentos de acordo com seu perfil, metas e objetivos.

\section{JOGOS SÉRIOS}
Jogos sérios são desenvolvidos para serem cativantes ao mesmo tempo em que alcançam algum objetivo pedagógico. Eles buscam causar um impacto nos seus jogadores, aprimorando sua experiência de aprendizagem e portanto devem ser tão atrativos quanto jogos de entretenimento enquanto ensinam \cite{bellotti2013}. Se bem sucedidos, eles podem ser um meio de aproximação da escola com a forma de pensar dos alunos. Através de uma participação ativa e não-linear que, além de ser uma linguagem mais próxima dos nativos digitais\footnote{Nativos Digitais: Termo cunhado por Marc Prensky,  para referenciar aos que nascem e crescem interagindo com as tecnologias digitais.}, eles permitem a simulação de situações que o aluno pode manipular de forma segura, com \textit{feedbacks} imediatos. Existe uma grande quantidade de pesquisas que descrevem os efeitos positivos da utilização de jogos na sala de aula. Embora existam críticas em relação ao métodos de avaliação do impacto de jogos na aprendizagem, Connolly et al (\citeyear{connolly2012}) identificou 129 artigos que reuniram evidências empíricas que jogos, digitais ou tradicionais de fato contribuem para a aquisição de conhecimento e outras habilidades. Nesses artigos foram encontrados em diferentes aplicações de jogos melhorias cognitivas, perceptivas, comportamentais, afetivas e motivacionais nos jogadores, assim como a aquisição e compreensão do conteúdo.

Buscar novas formas de transmitir o conhecimento em sala de aula se torna relevante quando se considera que os novos meios de interação que surgem influenciam as formas de pensar e consequentemente os estilos de aprendizagem dos alunos enquanto os meios tradicionais de ensino da escola se tornam cada vez mais obsoletos \cite{prensky2012}. Dessa forma, as próprias características motivacionais e engajantes que o jogo possuí se tornam relevantes para justificar sua aplicação em sala de aula. 

Ao mesmo tempo que um jogo sério deve ser engajante, ele não pode desviar de seu objetivo de ensinar. Segundo Piccini,
\begin{citacao}
A melhor forma de ensinar através de jogos é, então, fazer com que suas regras sejam semelhantes ao que o jogador precisa aprender e não forçando o jogador a aprender um jogo para ir sendo “informado” do conteúdo. Em outras palavras, é mais coerente que um jogo que se proponha a ensinar matemática estimule o jogador a solucionar problemas de soma, subtração, divisão e multiplicação ao invés de obrigar o jogador a aprender a controlar uma personagem que percorre um labirinto resgatando algarismos e sinais de operações aritméticas. (\citeauthor{piccini2008}, \citeyear{piccini2008}, pg.26)
\end{citacao}

Jogos são compostos de regras, que definem as ações permitidas ao jogador e as consequências que elas terão dentro do sistema apresentado. Para que a aquisição e compreensão do conteúdo ocorra é necessário que ele esteja entrelaçado nessas regras que compõem a base do jogo, de forma que somente ao dominar os conhecimentos que se busca ensinar o jogador seja capaz de manipular o sistema a seu favor. Portanto, o desenvolvimento de um jogo sério deve considerar o assunto que será ensinado como base para a criação do sistema de regras, buscando na teoria a ser ensinada as ações e consequências que aparecerão ao longo do jogo.
	
Dessa forma, para alcançarem seu potencial, é importante que os jogos sérios sejam desenvolvidos de forma a pensar no conteúdo que pretendem ensinar e aplicados em planos de aula planejados para sua utilização.

\chapter{METODOLOGIA DA PESQUISA}
A presente pesquisa terá uma abordagem qualitativa ao analisar os dados obtidos, que serão recolhidos por intermédio das respostas dos estudantes nos questionários em anexo. Segundo Minayo (\citeyear{minayo2008}) a pesquisa qualitativa possibilita reorganizar relações, procedimentos, símbolos e acepções da realidade social.

A educação financeira no Ensino Médio Integrado da RFEPCT é algo recente, ainda não estabelecido, o que classifica a pesquisa como exploratória quanto aos objetivos e aplicada em relação à sua natureza. Os estudos bibliográficos originarão um estudo de caso com um pressuposto teórico, que será validado por meio dos procedimentos e objetivos do projeto. Segundo YIN,
\begin{citacao}
Um estudo de caso é uma investigação empírica que investiga um fenômeno contemporâneo dentro de seu contexto da vida real, especialmente quando os limites entre o fenômeno e o contexto não estão claramente definidos.\cite{yin2005}.
\end{citacao}
Para atender aos objetivos primários realizou-se uma pesquisa exploratória sobre Educação Financeira, Jogos Sérios e Simuladores nos projetos pedagógicos dos cursos técnicos integrados nos Institutos Federais, para verificar a existência do ensino nos \textit{campi} em todo Brasil; nos principais repositórios de trabalhos acadêmicos, congresso, revistas e cursos voltados para jovens e adolescentes, para identificar as metodologias e práticas de ensino; assim como, nas lojas físicas ou virtuais de aplicativos, games e jogos, para encontrar tecnologias que possam ser utilizadas no ensino e aprendizagem da Educação Financeira para adolescentes.

Em seguida, elaborar-se-á um plano de aulas com o seguinte conteúdo programático:
\begin{itemize}
    \item Capital, Juros, Tabela Price e SAC;
    \item Nossa Relação com o Dinheiro e Orçamento Pessoal;
    \item Consumo Planejado e Consciente, Crédito e Endividamento;
    \item Serviços Bancários, Prevenção e Proteção;
    \item Poupança e Noções de Investimentos.
\end{itemize}

O plano de ensino foi aplicado em 8 encontros presenciais no Curso de Educação Financeira para Jovens do Ensino Médio Integrado do IFSUL - Câmpus Gravataí no segundo semestre de 2019. Este Projeto de Ensino foi aprovado pela Pró-reitoria de Ensino do IFSUL e fora ministrado por 3 colaboradores especialistas em Matemática Financeira, Contabilidade e Investimentos, Informática e Educação.

A infraestrutura necessária para o desenvolvimento desse projeto e execução do plano de ensino nas aulas inclui:
\begin{itemize}
    \item Material impresso e digital;
    \item Material auxiliar disponibilizado na plataforma Moodle;
    \item Laboratório de Informática;
    \item Jogos Digitais e Tabuleiro;
    \item Projetor multimídia;
    \item Quadro branco.
\end{itemize}

O conteúdo do curso foi ministrado através de exposições orais seguidas de aulas práticas com atividades no laboratório de informática e/ou trabalhos escritos, direcionados para a resolução de problemas e exercícios práticos de finanças pessoais e investimentos, visando fixar o conteúdo abordado, com auxílio de tecnologias educacionais como jogos, simuladores e ambientes virtuais de aprendizagem, entre outros. Acredita-se que buscar novas formas de transmitir o conhecimento em sala de aula é relevante, principalmente quando os novos meios de interação que surgem influenciam as formas de pensar e, consequentemente, os estilos de aprendizagem dos estudantes \cite{prensky2012}. Jogos Sérios são desenvolvidos para serem cativantes ao mesmo tempo em que se propõem a alcançar algum objetivo pedagógico.

\section{Os Jogos}
No decorrer do curso utilizou-se dois jogos como apoio à aprendizagem dos discentes, o Orçamento Consciente e o Renda Passiva, os aspectos inerentes ao jogo, bem como o uso destes nos processos educativos estão descritos neste capítulo.

\subsection{O Jogo Orçamento Consciente}
O Jogo Orçamento Consciente é um protótipo de um jogo sério digital, disponível para computadores e smartphones, criado em 2018 pelos alunos Camila Peres, Johnata Santicioli e Willian Chimura do Mestrado Profissional em Informática na Educação no Instituto Federal do Rio Grande do Sul.

O jogo simula o recebimento de um salário mensal de R\$3.000,00 que deve ser dividido em três necessidades: compras e conforto, alimentação e saúde nesta ordem da esquerda à direita conforme a Figura \ref{fig: figura01}, neste momento o jogador realiza um orçamento financeiro mesmo que de forma aleatória, após distribuir todo o salário do mês o jogo apresenta uma contagem regressiva para início do jogo, Figura \ref{fig: figura02}.

\graphicspath{{figuras/}}
\begin{figure}[!ht]
\center
\begin{minipage}{0.4\linewidth}
\center
\caption{Distribuição do salário} \label{fig: figura01}
\includegraphics[width=1.0\linewidth]{01-figura_distribuicao-salario.png}
\legend{\footnotesize Fonte: Dos Autores}
\end{minipage}
\begin{minipage}{0.4\linewidth}
\center
\caption{Contagem regressiva} \label{fig: figura02}
\includegraphics[width=1.0\linewidth]{02-figura_contagem-regressiva-mes.png}
\legend{\footnotesize Fonte: Dos Autores}
\end{minipage}
\end{figure}

Ao longo de “um mês” dentro do jogo, que dura aproximadamente 1 minuto, o jogador deve equilibrar suas necessidades, escolhendo quais produtos comprar e quais ignorar, considerando o custo-benefício e seu planejamento inicial (Figura \ref{fig: figura03} e \ref{fig: figura04}), podendo apenas utilizar o dinheiro da necessidade com o respectivo produto, em outras palavras, utilizar o dinheiro reservado para saúde apenas com produtos de saúde. O jogador não pode transferir dinheiro de uma necessidade para outra, ao esgotar os recursos destinados para determinada necessidade, o jogador precisa esperar até o mês seguinte para repensar sua estratégia e ajustar o planejamento orçamentário para torná-lo superavitário ou neutro.

\graphicspath{{figuras/}}
\begin{figure}[!ht]
\center
\begin{minipage}{0.4\linewidth}
\centering
\caption{Produto mercado} \label{fig: figura03}
\includegraphics[width=1.0\linewidth]{03-figura_tela-jogo-mercado.png}
\legend{\footnotesize Fonte: Dos Autores}
\end{minipage}
\begin{minipage}{0.4\linewidth}
\centering
\caption{Produto shopping} \label{fig: figura04}
\includegraphics[width=1.0\linewidth]{04-figura_tela-jogo-shopping.png}
\legend{\footnotesize Fonte: Dos Autores}
\end{minipage}
\end{figure}

A tela principal do jogo, exibidas nas  Figura \ref{fig: figura03} e \ref{fig: figura04}, apresenta três “prédios" (construções do jogo), um azul que representa um shopping, relacionado à “necessidade de compras e conforto”, um mercado na cor laranja, que corresponde a “necessidade de comida” e uma farmácia rosa simbolizando a “necessidade de remédios ou saúde”. As barras horizontais vermelhas (“barras de necessidade”) embaixo de cada construção indicam ao jogador a necessidade de consumir produtos de acordo com sua categoria. Essas barras iniciam totalmente preenchidas (necessidade de consumo atendida) e diminui conforme o tempo do jogo, representando o dever de uma nova compra. Abaixo de cada uma das barras encontra-se a quantidade de dinheiro distribuída no começo de cada mês do jogo e disponível para o consumo da respectiva necessidade.

Cada “prédio” oferece opções de produtos relativos à que representa, um após o outro, cabe ao jogador a decisão de comprar ou não os produtos disponíveis. Para efetuar a compra o jogador deve selecionar o produto, considerando o planejamento orçamentário realizado, se o item é necessário e se é viável a sua aquisição. Cada produto possui diferentes níveis de qualidade, representados por estrelas (de uma a três), bem como diferentes preços, representados por cifrões (de uma a três), cada um equivale a cem reais. Quanto maior a qualidade do produto, maior será a quantidade de estrelas e supre mais a necessidade de consumo, aumentando o preenchimento da barra vermelha que indica a necessidade. Por exemplo, a Figura \ref{fig: figura03} ilustra um item que pode ser comprado pelo o jogador junto ao prédio do supermercado, um cacho uva que possui qualidade nível 1, pois tem uma estrela, e preço de R\$ 200 (duzentos reais), porque há 2 (dois) cifrões. Cada item comprado faz com que o orçamento planejado sofra alterações, diminuindo os recursos para a aquisição de itens de uma mesma categoria. Os produtos ficam disponíveis para compra por um determinado tempo, controlado pelo círculo laranja em volta do produto, quando o círculo completa uma volta o tempo para comprar o produto acaba (uso dos Gatilhos Mentais da Urgência e Escassez), então desaparece e eventualmente são substituídos por outros, com diferentes qualidades e preços. Nas figuras \ref{fig: figura03} e \ref{fig: figura04}, acima das construções existe um contador de estrelas, referentes a qualidade do produto. Ao comprar produtos de maior qualidade essa pontuação aumenta conforme a quantidade de estrelas do produto, ao esvaziar completamente a barra de necessidade, essa ficará cinza, pois não houve consumo suficiente. Cabe ao jogador escolher quais produtos comprar, baseando-se em seu planejamento inicial, preço, qualidade e necessidade.

Ao final do mês o jogador recebe um relatório detalhando as necessidades que foram atendidas ou não e sua pontuação (Figuras \ref{fig: figura05} e \ref{fig: figura06}), a cada dia que uma das barras de necessidade estiver totalmente vazia é descontado uma estrela do total, em caso de sobra de recursos, cada R\$ 100 é convertido em uma estrela. Por exemplo na Figura \ref{fig: figura06}, neste relatório o jogador passou 4 dias sem recurso no Shopping, 3 dias sem recursos no Mercado e poupou com R\$ 500 de seu salário, com isso duas estrelas serão retiradas das 56 que ele possui.

\graphicspath{{figuras/}}
\begin{figure}[!ht]
\center
\begin{minipage}{0.4\linewidth}
\center
\caption{Tela de relatório 1} \label{fig: figura05}
\includegraphics[width=1.0\linewidth]{05-figura_relatorio-mes-1}
\legend{\footnotesize Fonte: Dos Autores}
\end{minipage}
\begin{minipage}{0.4\linewidth}
\center
\caption{Tela de relatório 2} \label{fig: figura06}
\includegraphics[width=1.0\linewidth]{06-figura_tela-relatorio-mes-2}
\legend{\footnotesize Fonte: Dos Autores}
\end{minipage}
\end{figure}

Dessa forma, o jogo permite aos alunos a possibilidade de testar os conhecimentos teóricos em uma simulação simples, podendo ver as consequências de suas ações de forma rápida em um ambiente seguro.

\subsection{O Jogo Renda Passiva}
O Jogo Renda Passiva é um jogo sério de tabuleiro desenvolvido por Daniel Frechiani, diferente dos jogos tradicionais de tabuleiro onde a sorte é condição principal para ganhar, no Renda Passiva os jogadores deverão desenvolver a capacidade de resolver problemas financeiros para ter êxito no jogo \cite{frechiani2019a}, de uma forma lúdica, divertida e inteligente pode-se aprender conceitos da Educação Financeira através do jogo.

O jogo simula a vida financeira de um personagem podendo o jogador escolher entre um microempresário, servidor público, empresário, motorista de aplicativo, professora e advogada. Todos os personagens possuem Receitas (salário ou pró-labore), Gastos (Fixos, Variáveis, Impostos, Seguros, Financiamento, Cartão de Crédito e Diversos), Dívidas e podem obter no decorrer do jogo Ativos (Ações e Títulos de Renda Fixa) conquistando assim uma renda passiva. O controle e anotações destes itens devem ser feitos através da Ficha Financeira (Figura \ref{fig: figura07}). Além das fichas o jogo possui cartas de negócio, imóvel, renda fixa, oferta, ações, cartas surpresas, dados para percorrer as casas, pinos que representam os jogadores, canetas e apagadores para utilizar as fichas financeiras.

\graphicspath{{figuras/}}
\begin{figure}[!ht]
\centering
\begin{minipage}{1.\textwidth}
\caption{Ficha Financeira}
\centering
\includegraphics[width=0.5\textwidth]{07-figura_ficha-financeira-renda-passiva}
\legend{\footnotesize Fonte: Frechiane, Manual do Jogo (\citeyear{frechiani2019b})}
\label{fig: figura07}
\end{minipage}
\end{figure}

Ao iniciar o jogo os jogadores percorrem tabuleiro (Figura \ref{fig: figura08}) e precisam passar pelos ícones de sacos de dinheiros para receberem pagamentos (já com os gastos descontados).

\graphicspath{{figuras/}}
\begin{figure}[!ht]
\centering
\begin{minipage}{1.\textwidth}
\caption{Tabuleiro do Jogo}
\centering
\includegraphics[width=0.8\textwidth]{08-figura_tabuleiro-renda-passiva}
\legend{\footnotesize Fonte: Frechiane, Manual do Jogo (\citeyear{frechiani2019b})}
\label{fig: figura08}
\end{minipage}
\end{figure}

Diferente da maior parte dos jogos sobre finanças que possui um caminho único, neste o jogador pode decidir por qual caminho seguir no tabuleiro de acordo com o planejamento que possui, assim determinar uma estratégia de jogo, sendo capaz de decidir entre quitar as dívidas, investir em ações, renda fixa, comprar imóveis ou negócios, considerando o risco, a liquidez e a rentabilidade de cada um. Para isso basta apenas movimentar-se para os respectivos cenários (Figura \ref{fig: figura09}) e retirar uma carta, exceto as cartas ações que podem ser solicitados de qualquer posição do tabuleiro.

\graphicspath{{figuras/}}
\begin{figure}[!ht]
\centering
\begin{minipage}{1.\textwidth}
\caption{Cenários do Jogo}
\centering
\includegraphics[width=0.6\textwidth]{09-figura_cenario-jogo-renda-passiva}
\legend{\footnotesize Fonte: Frechiane, Manual do Jogo (\citeyear{frechiani2019b})}
\label{fig: figura09}
\end{minipage}
\end{figure}

Nas cartas surpresa o jogador pode encontrar riscos e oportunidades como: demissão, crise econômica, adição de um novo filho à família (aumentando o gasto), 13º Salário, promoção no trabalho, recebimento de dividendos entre outros.

O jogador que conseguir cumprir 3 objetivos: quitar todas as dívidas, ter uma renda passiva igual ou maior ao seu total de gastos, chegar ao cenário da conquista (casa com cadeado), vence o jogo.

\section{A Avaliação}
Executou-se análise dos dados por intermédio de uma abordagem qualitativa, com o objetivo de verificar se as metodologias adotadas no decorrer do Curso de Educação Financeira contribuíram no processo de aprendizagem, bem como aferir o nível retenção dos temas principais apresentados, conhecimento prévio, motivação para participar do curso, o interesse em buscar novos conhecimentos, a disseminação do conhecimento adquirido e a consciência da importância do tema.

Os dados foram coletados através de formulários virtuais na derradeira etapa do curso, exceto os para avaliar a experiência dos alunos e sua percepção de aprendizagem em relação ao Jogo Sério, coletados logo após a utilização dos jogos. As informações obtidas por meio do Questionário de Avaliação da Aprendizagem (Apêndice A), Questionário de Avaliação do Curso (Apêndice B) e pelo método avaliativo MEEGA+ (\textit{Model for Evaluating Educational Games}), possibilitaram aos pesquisadores categorizar e analisar as respostas no intuito de verificar quais objetivos propostos foram atingidos e a confirmação ou não da hipótese.

O método MEEGA+ possui um questionário para avaliar a experiência subjetiva do jogador, permitindo a coleta de dados sobre aspectos como o foco da atenção, diversão, desafio, interações sociais, relevância, satisfação, usabilidade e percepção de aprendizagem \cite{petri2017}, também apresenta uma seção para relatar pontos fortes, fracos e comentários adicionais. Este busca agir de forma rápida e não intrusiva para maior efetividade em avaliar a qualidade dos jogos.

Ao refletir sobre os dados coletados, espera-se chegar a conclusão sobre a potencialidade que os jogo possibilitam para aquisição do conhecimento relativo aos conteúdos abordados.

\chapter{ANÁLISE DOS TRABALHOS RELACIONADOS}
No intuito de averiguar a utilização de jogos como apoio à Educação Financeira em metodologias apuradas no ensino médio integrado no Brasil, bem como, para encontrar possíveis demandas, propor e justificar a elaboração do presente trabalho e de seu produto, buscou-se encontrar estudos acadêmicos nas principais bases de teses e dissertações realizados entre 2010 e 2020. Fora encontrado alguns trabalhos relacionados à Matemática Financeira e poucos acerca da Educação Financeira, entretanto, a Alfabetização Financeira crítica não é alcançada apenas com o conhecimento de fórmula e cálculos matemáticos, por isso, procurou-se embasar a pesquisa nos trabalhos voltados à Educação Financeira.

Foram analisados nesta pesquisa seis dissertações de mestrado os quais assemelham-se ao presente trabalho e estão descritos no quadro \ref{quad: quadro03}.

\graphicspath{{quadros/}} 
\begin{quadro}[!ht]
\centering
\begin{minipage}{1.\textwidth}
\caption{Trabalhos Relacionados}
\centering
\includegraphics[width=1.0\textwidth]{quadro-3_trabalhos}
\legend{\footnotesize Fonte: O autor, 2019}
\label{quad: quadro03}
\end{minipage}
\end{quadro}

A dissertação de Gelson Pietras (\citeyear{pietras2014}), propõem uma abordagem para apresentação de alguns assuntos referente a educação financeira para o ensino médio, entre os quais podemos destacar: matemática financeira, desejos, hábitos de consumo, poupança, dívidas, compras, capitalização e investimentos, por intermédio de uma sequência de atividades práticas com cálculos de juros entre outros, em uma destas utilizando o Jogo “Banco Imobiliário” no processo de aprendizagem para que os alunos possam colocar em prática os conhecimentos adquiridos de forma lúdica. O autor concluiu que as atividades trouxeram ponderação sobre os hábitos financeiros mais comuns o que auxiliará nas próximas decisões de compra do alunos.

Com o Título “Matemática Financeira no Ensino Médio: um Jogo para Simulação”, Renato Schneider Rivero Jover (\citeyear{jover2014}) tem o intuito de promover a Educação Financeira de estudantes do Ensino Médio com uma metodologia baseada no Jogo “Investindo na Vida”, com a fundamentação teórica financeira apresentada por Robert Kiyosaki e concepções pedagógicas de Maria Montessori, Ovide, Décroly, Lev Vygotsky, Jean Piaget e Donald Winnicott, também apresenta alguns artigos para enfatizar a importância do lúdico na educação. A metodologia  aborda o consumo, operações do mercado, ações e investimentos. Assim como, aponta a importância da Educação Financeira para a formação social do indivíduo e a utilização jogos para auxiliar na aprendizagem.

A pesquisa apresentada por Flávia de Oliveira Carvalho (\citeyear{flavia2016}) em sua dissertação de mestrado,  “Desafios Financeiros: o Papel do Design de um Jogo de Educação Financeira”, tem objetivo de analisar o processo de construção do Jogo “Desafio Financeiros” como material pedagógico no ensino médio, visando contribuir na formação cidadã-crítica do jovens em relação aos hábitos de consumo destes. No resultado do trabalho demonstra que o jogo contribuiu no processo de aprendizagem dos alunos.

Eduardo Wartchow (\citeyear{wartchow2017}), em sua dissertação intitulada “EduCoelho: uma plataforma de alfabetização financeira estimulando a reflexão comportamental”, descreve que a Alfabetização Financeira, processo no qual o indivíduo alcança equilíbrio em sua vida financeira, pode ser alcançada por intermédio da Educação Financeira. O trabalho utiliza o Jogo “EduCoelho” e um canal no \textit{Youtube} como de recurso tecnológico educacional para facilitar a compreensão e a aprendizagem dos alunos de um assunto tão complexo. O Autor concluiu que a metodologia utilizada pode contribuir para provocar o reflexão comportamental nos alunos a respeito da educação financeira. 

Jonatas Campos Sarlo (\citeyear{sarlo2019}) apresenta um estudo sobre a educação financeira e propõe atividades pedagógicas com recursos diferenciados como teatro, simulações de situações-problemas e jogos eletrônicos, como o “Jogo da Bolsa” com objetivo de auxiliar no desenvolvimento da autonomia do aluno em suas escolhas financeiras, bem como enriquecer as prática pedagógica em metodologias de ensino e aprendizagem da Educação Financeira. Em suas considerações finais Sarlo (\citeyear{sarlo2019}) conta que o objetivo de sua pesquisa foi alcançado com a metodologia desenvolvida, também aponta o interesse dos alunos para a temáticas das aulas principalmente  quando estas envolviam tecnologias digitais.

Durante a pesquisa por trabalhos relacionados encontrou-se muito trabalhos acerca da Matemática Financeira os quais apresentaram especificamente cálculos financeiros, não obstante, estes foram descartados, pois, a formação financeira crítica-cidadã não é formada apenas pelo conhecimento de fórmulas matemáticas. Entretanto fora escolhido o trabalho “Entendendo a matemática financeira por meio de um material didático: o caso de uma turma do terceiro ano de Ensino Médio da Escola Wallace Castelo Dutra em São Mateus - ES” para fundamentar a unidade introdutória a matemática financeira e o descarte dos demais.

O estudo desenvolvido por Almir Rogério Luppi (\citeyear{luppi2018}) apresenta conceitos básicos do Sistema Financeiro e os conteúdos oficiais da educação brasileira que focam nos conteúdos da Matemática Financeira, por meio de Tecnologias da Informática e Comunicação (TICs), jogos e dinâmica em grupos. O trabalho aponta o juros compostos e sistema de amortização como os tópicos mais comuns, normalmente, ensinados com o auxílio da calculadora HP 12C, planilhas eletrônicas ou em softwares matemáticos como o Geogebra. Na pesquisa o autor utiliza-se do Jogo “Super Banco Imobiliário” para colocar em prática de forma lúdica alguns conceitos. Luppi (\citeyear{luppi2018}) afirma que a metodologia proporcionou noções importantes da Matemática Financeira aos alunos participantes e contribuiu para a Estratégia Nacional de Educação Financeira.

\chapter{PRODUTO: CURSO DE INTRODUÇÃO À EDUCAÇÃO FINANCEIRA}
Conforme citado em seções anteriores, o produto desta pesquisa é o Curso de Introdução à Educação Financeira para o Ensino Médio Integrado, voltado principalmente para os Institutos Federais e outras Autarquias da RFEPCT. Com  base nos resultados, sugestões e as propostas de trabalhos diretamente relacionados ao tema pesquisado foi possível estabelecer 5 unidades temáticas abordadas no curso, Matemática Financeira, Psicologia Comportamental Financeira, Consumo Consciente, Instituições Financeiras e Serviços, Investimentos e Previdência.

O conteúdo abordado no curso é ministrado em 8 encontros com 2 aulas de 45 minutos cada, o Quadro \ref{quad: quadro04} elenca as unidades temáticas em conjunto das respectivas ementa. O objetivo instrucional, conhecimento prévio, recursos utilizados, tema e desenvolvimento de cada encontro são descritos no decorrer deste capítulo.

\graphicspath{{quadros/}} 
\begin{quadro}[!ht]
\centering
\begin{minipage}{1.\textwidth}
\caption{Unidade Temáticas e Conteúdos}
\centering
\includegraphics[width=1.0\textwidth]{quadro-4_tematica}
\legend{\footnotesize Fonte: O autor, 2019}
\label{quad: quadro04}
\end{minipage}
\end{quadro}

\section{1º Encontro - Matemática Financeira}
Quando possível, o primeiro encontro deve ser ministrado por um docente especialista em matemática financeira e educação, pois este, teoricamente, detém um domínio amplo sobre matemática financeira. Partindo do príncipio que os alunos têm conhecimento básico sobre o Sistema Monetário Brasileiro para efetuar contratação de serviços ou produtos ensinado a estes nos anos iniciais do ensino fundamental, o Quadro \ref{quad: quadro05} apresenta o Plano de Aula do 1º encontro no qual é abordado especificamente a Unidade Temática Introdução à Matemática financeira.

\newpage

\graphicspath{{quadros/}} 
\begin{quadro}[!ht]
\centering
\begin{minipage}{1.\textwidth}
\caption{Plano de Aula 1º Encontro}
\centering
\includegraphics[width=0.9\textwidth]{quadro-5_plano-aula-1}
\legend{\footnotesize Fonte: O autor, 2019}
\label{quad: quadro05}
\end{minipage}
\end{quadro}

Para realizar os cálculos financeiros durante a aula, são utilizados exemplos de produtos os quais os façam parte do conhecimento dos discentes para tornar o processo de aprendizagem interessante e engajante.

\section{2º Encontro - Psicologia Comportamental}
A unidade temática do segundo encontro é a Introdução da Psicologia Financeira comportamental, com os seus respectivos conteúdos apresentados no Quadro \ref{quad: quadro06} que apresenta o Plano de Aula do 2º encontro.

\newpage

\graphicspath{{quadros/}} 
\begin{quadro}[!ht]
\centering
\begin{minipage}{1.\textwidth}
\caption{Plano de Aula 2º Encontro}
\centering
\includegraphics[width=0.9\textwidth]{quadro-6_plano-aula-2}
\legend{\footnotesize Fonte: O autor, 2019}
\label{quad: quadro06}
\end{minipage}
\end{quadro}

O Jogo Sério é utilizado por intermédio do computador onde cada discente interage individualmente com o \textit{game}, desta forma pode-se observar, associar e obter melhor compreensão sobre os temas abordados em aula. No final da aula é aplicado uma avaliação do jogo por intermédio do método MEEGA+.

Estruturou-se a presente aula para auxiliar no processo de ensino e aprendizagem, bem como no desenvolvimento cognitivo do estudante, para que este inicie a transição do estágio \textbf{Consumidor Sensorial} (consumo baseado em sensações) para o \textbf{Consumidor Pré-Poupador} (consumo consciente dos recursos financeiros), assim como assimilar a importância do Planejamento Orçamentário.

\section{3º Encontro - Consumo Consciente}
Após os discentes entenderem a importância de planejar seus gastos, a unidade temática do terceiro encontro apresenta o Consumo Consciente relacionado às finanças, que possui o intuito de demonstrar como executar o Planejamento Orçamentário. Os conteúdos relacionados a unidade temática estão elencados no Quadro \ref{quad: quadro07} que apresenta o Plano de Aula do 3º encontro.

\graphicspath{{quadros/}} 
\begin{quadro}[!ht]
\centering
\begin{minipage}{1.\textwidth}
\caption{Plano de Aula 3º Encontro}
\centering
\includegraphics[width=0.9\textwidth]{quadro-7_plano-aula-3}
\legend{\footnotesize Fonte: O autor, 2019}
\label{quad: quadro07}
\end{minipage}
\end{quadro}

Após apresentação do conteúdo os discente realizam a Atividade de Planejamento Orçamentário, individual (Pessoal) ou em dupla (Familiar), na qual definirão uma profissão que almejam trabalhar, buscar o salário médio respectivo e defini-lo como receita, o plano deve conter despesas com habitação (aluguel e condomínio), saúde, transporte, educação, contas residenciais e alimentação, o professor define um valor por pessoa por cada categoria de gastos.

Os alunos deverão poupar valores em seus planos e retirar uma carta surpresa do Jogo Renda Passiva que influenciará no orçamento. As cartas surpresa do Jogo contêm oportunidades (restituição de imposto, bônus da empresa, promoção no trabalho etc.), riscos (multas de trânsito, demissão, imposto etc) entre outros como a chegada de um filho que aumenta a despesa em 10\% do salário, Figura \ref{fig: 10-figura}.

\graphicspath{{figuras/}} 
\begin{figure}[!ht]
\centering
\begin{minipage}{1.\textwidth}
\caption{Carta Surpresa}
\centering
\includegraphics[width=0.3\textwidth]{10-figura_carta-surpresa-renda-passiva}
\legend{\footnotesize Fonte: Frechiane (\citeyear{frechiani2019b})}
\label{fig: 10-figura}
\end{minipage}
\end{figure}

Ao aplicar o conhecimento adquirido nessa aula em sua vida, o aluno alcança o estágio \textbf{Consumidor Pré-Poupador} do desenvolvimento cognitivo financeiro, consumindo de forma consciente com planos de aquisição de produto e serviços pré estabelecidos, diminuindo o impacto das Estratégias de Vendas e Gatilhos Mentais sem suas decisões de compras.

\section{4º Encontro - Instituições Financeiras}
O objetivo do quarto encontro é explanar aos estudantes os tipos de Insitiuições Financeiras (Bancos,Corretoras de Investimento, Seguradoras e Financeiras) e seus Respectivos Serviços (Cesta de Serviço, Seguros, Investimentos, Crédito e Portabilidade ), bem como sobre dívidas e endividamentos. Os conteúdos relacionados a unidade temática estão descritos no Quadro \ref{quad: quadro08} que apresenta o Plano de Aula do 4º encontro.

\newpage
\graphicspath{{quadros/}} 
\begin{quadro}[!ht]
\centering
\begin{minipage}{1.\textwidth}
\caption{Plano de Aula 4º Encontro}
\centering
\includegraphics[width=0.9\textwidth]{quadro-8_plano-aula-4}
\legend{\footnotesize Fonte: O autor, 2019}
\label{quad: quadro08}
\end{minipage}
\end{quadro}

Ao final deste encontro aplicando o conhecimento adquirido o aluno alcança o estágio \textbf{Poupador Concreto} desenvolvimento cognitivo financeiro, pois compreende a importância e aplicam o Planejamento Orçamentário, buscam sempre poupar recursos, realizam Consumo Consciente, conhecem os tipos de serviços e instituições financeiros, assim como investem o recurso poupado na caderneta de poupança, por falta de conhecimento em investimentos.

\section{5º Encontro - Introdução aos Investimentos}
O objetivo do quinto encontro é demonstrar aos estudantes os aspectos relacionados à Introdução de Investimentos Financeiros, bem como iniciar a transição do desenvolvimento cognitivo financeiro para o último estágio (\textbf{Investidor Formal}). Os conteúdos relacionados a unidade temática estão descritos no Quadro \ref{quad: quadro09} que apresenta o Plano de Aula do 5º encontro.

\graphicspath{{quadros/}} 
\begin{quadro}[!ht]
\centering
\begin{minipage}{1.\textwidth}
\caption{Plano de Aula 5º Encontro}
\centering
\includegraphics[width=1.0\textwidth]{quadro-9_plano-aula-5}
\legend{\footnotesize Fonte: O autor, 2019}
\label{quad: quadro09}
\end{minipage}
\end{quadro}

Neste encontro ocorre a iniciação do Jogo Renda Passiva com explanação de suas cartas, regras, tabuleiro, fichas, pins e jogabilidade, por trata-se de um jogo novo com muitas regras é possível que alguns alunos tenham dificuldades iniciais, entretanto o jogo simula de forma lúdica os conhecimentos adquiridos no curso e do mercado financeiro o que auxilia os alunos na fixação dos conteúdos.

\section{6º Encontro - Revisão com o Jogo Renda Passiva}
O objetivo do sexto encontro é revisar junto com os discentes os conteúdos abordados no curso (Planejamento Orçamentário, Portabilidade de Crédito e Investimentos), bem como a acomodação dos conteúdos estudados através de um Jogo Sério. O Plano de Aula deste encontro está descrito no Quadro \ref{quad: quadro10}.

\graphicspath{{quadros/}} 
\begin{quadro}[!ht]
\centering
\begin{minipage}{1.\textwidth}
\caption{Plano de Aula 6º Encontro}
\centering
\includegraphics[width=0.9\textwidth]{quadro-10_plano-aula-6}
\legend{\footnotesize Fonte: O autor, 2019}
\label{quad: quadro10}
\end{minipage}
\end{quadro}

\section{7º Encontro - Renda Variável e Previdência}
O objetivo do sétimo encontro é apresentar aos estudantes os aspectos relacionados à Introdução de Investimentos Financeiros especificamente em Renda Variável, bem como explicar o funcionamento e os tipos de Previdência no Brasil. Os conteúdos relacionados a unidade temática estão descritos no Quadro \ref{quad: quadro11} que apresenta o Plano de Aula do 7º encontro.

\graphicspath{{quadros/}} 
\begin{quadro}[!ht]
\centering
\begin{minipage}{1.\textwidth}
\caption{Plano de Aula 7º Encontro}
\centering
\includegraphics[width=0.9\textwidth]{quadro-11_plano-aula-7}
\legend{\footnotesize Fonte: O autor, 2019}
\label{quad: quadro11}
\end{minipage}
\end{quadro}

Ao final deste encontro aplicando o conhecimento adquirido e buscando estudar novos conteúdos sobre finanças, o aluno poderá alcançar o estágio \textbf{Investidor Formal} desenvolvimento cognitivo financeiro, pois compreende a importância e aplicam o Planejamento Orçamentário, buscam sempre poupar recursos, realizam Consumo Consciente, conhecem os tipos de serviços e instituições financeiros, assim como investem o recursos em diferentes instituições de acordo com o seu perfil e objetivos financeiros.

\section{8º Encontro - Jogo Renda Passiva e Aplicação dos Questionários}
O objetivo do oitavo encontro é revisar junto com os discentes todos os conteúdos abordados no curso através de um Jogo Sério, entretanto desta vez os alunos utilizaram o Jogo Renda Passiva com todos os recursos e regras disponíveis. O Plano de Aula deste encontro está descrito no Quadro \ref{quad: quadro12}.

\graphicspath{{quadros/}} 
\begin{quadro}[!ht]
\centering
\begin{minipage}{1.\textwidth}
\caption{Plano de Aula 8º Encontro}
\centering
\includegraphics[width=0.9\textwidth]{quadro-12_plano-aula-8}
\legend{\footnotesize Fonte: O autor, 2019}
\label{quad: quadro12}
\end{minipage}
\end{quadro}

Após a utilização do Jogo Sério é aplicado uma avaliação para compreender experiência do jogador, tal como o processo de aprendizagem por intermédio do método MEEGA+.

Aplicam-se também o Questionário de Avaliação da Aprendizagem (Anexo I) e o Questionário de Avaliação do Curso (Anexo II), para verificar se os objetivos proposto foram alcançados e confirmar ou não a hipótese do presente trabalho.

\chapter{RESULTADOS PARCIAIS}
Curso de Educação Financeira para o Ensino Médio Integrado oferecido por intermédio de um projeto de ensino no Ifsul-câmpus Gravataí é uma disciplina não obrigatória, por isso as inscrição de 41 alunos, surpreendeu de forma positiva, pois representa o interesse pelo assunto de 15\% dos alunos matriculado no ensino médio da instituição, entretanto os recursos disponíveis para o curso não suportava essa quantidade de inscritos, por isso houve a necessidade de realizar um sorteio para o preenchimento das 19 vagas do curso.

O Curso ocorreu de 14/08/2019 a 27/11/2019 com duas aulas semanais. Os alunos do curso tiveram uma frequência superior a 75\%, exceto uma aluna que não permaneceu até o final, pois solicitou a transferência para outra instituição, fato que indica o engajamento dos alunos no processo de aprendizagem, tendo em vista que os estudantes não eram obrigados a permanecer no curso, pois este não faz parte do projeto pedagógico do ensino médio da instituição. Com isso os dados apresentados como resultados parciais serão apresentados nas interações dos 18 alunos com as atividades, tarefas e questionários do curso.

A avaliação do presente projeto baseia-se em 4 instrumentos (Questionário de Avaliação da Aprendizagem, Questionário de Avaliação do Curso e avaliação do resultado da interação com os jogos com o método MEEGA+, entretanto por ora será descrito os resultados da experiência subjetiva do jogador com o Jogo Orçamento Consciente para avaliar seu potencial de aprendizagem, para isto aplicou-se o questionário do MEEGA+ logo após a interação com os alunos com o Jogo Sério. As perguntas do questionário e os dados coletados com as respostas dos discentes são expostos no Gráfico \ref{graf: grafico01_meega}, bem como os pontos fortes do jogo apontados pelos alunos no Quadro \ref{quad: quadro13}.

\graphicspath{{graficos/}} 
\begin{grafico}[!ht]
\centering
%\begin{minipage}{1.\textwidth}
\caption{Avaliação do Jogo Orçamento Consciente}
\centering
\includegraphics[width=1.0\textwidth]{01-grafico_MEEGA+ Orçamento Consciente}
\legend{\footnotesize Fonte: Dos autores. Adaptado de Petri (\citeyear{petri2017})}
\label{graf: grafico01_meega}
%\end{minipage}
\end{grafico}

\newpage
Os resultados acerca da experiência do jogador revelam que os aspectos necessários para melhoria do jogo estão relacionados ao “Desafio”, “Interação social” e à “Atenção focada” (Gráfico \ref{graf: grafico01_meega}), abrindo margem para a hipótese de que a falta de incentivo à interação entre os jogadores, bem como de uma progressão de desafios mais adequada pode ter impedido com que houvesse um maior engajamento por parte dos alunos. Uma possível interpretação destes resultados pode sugerir que a atenção dos jogadores reduziu à medida que aprenderam a utilizar da forma lenta de pensar para cumprir com os objetivos, tornando o restante do jogo pouco desafiador para alguns.

\graphicspath{{quadros/}} 
\begin{quadro}[!ht]
\centering
\begin{minipage}{1.\textwidth}
\caption{Pontos Fortes - Jogo Orçamento Consciente}
\centering
\includegraphics[width=1.0\textwidth]{quadro-13_consumo-consciente-pontos-fortes}
\legend{\footnotesize Fonte: O autor, 2019}
\label{quad: quadro13}
\end{minipage}
\end{quadro}  

Na seção de pontos fortes do questionário MEEGA+, nota-se relatos acerca do pensamento rápido, gatilhos mentais, comportamento financeiro e compreensão da disciplina como pontos fortes do jogo. Os relatos dos alunos no Quadro \ref{quad: quadro13} e na “Percepção de Aprendizagem” (Gráfico \ref{graf: grafico01_meega}) confirmam a hipótese que a utilização de Tecnologias Educacionais como Jogos Sérios facilitam o processo de aprendizagem e satisfação dos discentes, assim como, sugerem que ideia dos desenvolvedores do jogo foi adequadamente expressa e percebida pelos jogadores. Tendo em vista que a ideia norteadora durante o desenvolvimento foi produzir um ambiente no qual o jogador precisasse lidar com sua forma rápida de pensar e passasse a pensar de forma lenta a fim de se obter mais recompensas.


% ==============================================
% ELEMENTOS PÓS-TEXTUAIS
% ==============================================
\postextual
% ||||||||||||||||||||||||||||||||||||||||||||||
% REFERÊNCIAS BIBLIOGRÁFICAS
% ||||||||||||||||||||||||||||||||||||||||||||||
%\bibliography{bibfile2}
% bibliografia john
\bibliography{ref-bibliografica}
% ----------------------------------------------
% Glossário
% ----------------------------------------------
% Consulte o manual da classe abntex2 para orientações sobre o glossário.
%
%\glossary
% ||||||||||||||||||||||||||||||||||||||||||||||
% APÊNDICES
% ||||||||||||||||||||||||||||||||||||||||||||||
\begin{apendicesenv}
% Imprime uma página indicando o início dos apêndices
\partapendices
% ----------------------------------------------
% Apêndice
% ----------------------------------------------
\includepdf[pages=-]{apendices/apendice-a_questionario-de-avaliacao-de-aprendizagem.pdf}
\includepdf[pages=-]{apendices/apendice-b_questionario-de-avaliacao-do-curso.pdf}
\end{apendicesenv}
% ||||||||||||||||||||||||||||||||||||||||||||||
% ANEXOS
% ||||||||||||||||||||||||||||||||||||||||||||||
%\begin{anexosenv}
% Imprime uma página indicando o início dos anexos
%\partanexos
% ----------------------------------------------
% Anexo 1
% ----------------------------------------------
%\chapter{Datasheet}\label{anexo1}
%\includepdf[pages=-]{pdfs/Datasheet.pdf}
% ----------------------------------------------
% Anexo 2
% ----------------------------------------------
%\chapter{Anexo 2}
%\end{anexosenv}
% ==============================================
% INDICE REMISSIVO
% ==============================================
\phantompart
\printindex
% ----------------------------------------------
% ||||||||||||||||||||||||||||||||||||||||||||||
% DOCUMENTO PARA ENTREGA VERSÃO FINAL
%||||||||||||||||||||||||||||||||||||||||||||||
\begin{comment}

\clearpage\thispagestyle{empty}\addtocounter{page}{-1}
\section*{ENTREGA DA VERSÃO FINAL DE DISSERTAÇÃO}

\vspace*{2cm}
Eu, \textsc{\imprimirorientador}, autorizo o aluno(a) \textsc{\imprimirautor} a entregar a versão final da dissertação de mestrado, à secretaria do MPIE, que foi por mim analisada e está de acordo com os apontamentos feitos pelos membros da banca de apresentação do referido aluno.

\vspace*{1cm}
\begin{center}   
   \assinatura{{\imprimirorientador} \\ Orientador}
\end{center}

\vspace*{1cm}
\begin{flushright}
	\imprimirlocal, 11 de Julho de \imprimirdata.
\end{flushright}
\clearpage
\end{comment}

\end{document}
