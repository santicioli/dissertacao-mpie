\begin{enumerate}
    \item O que é a educação financeira?
    
    \item Qual as características de um bom orçamento financeiro?
    
    \item Considerando a Educação Financeira e a Matemática Financeira assinale a alternativa correta:
        \begin{enumerate}
            \item  (   ) A Educação Financeira promove reflexão sobre a vida financeira.
           \item   (   ) A Matemática Financeira trata sobre hábitos financeiros.
            \item  (   ) A Matemática Financeira aborda os tipos de investimentos.
            \item  (   ) A Educação Financeira e a Matemática Financeira são sinônimos.
        \end{enumerate}
    
    \item Selecione a alternativa correta. O orçamento pessoal ou familiar auxilia:
        \begin{enumerate} 
            \item (   ) No controle de receitas e despesas
            \item (   ) Administrar os recursos financeiros de forma consciente
            \item (   ) No controle das dívidas
            \item (   ) Todas as alternativas anteriores
        \end{enumerate}
        
    \item Abaixo há uma lista de itens relacionados às necessidades, desejos e desperdício. Assinale as alternativas referentes somente às necessidades, pode-se escolher mais de uma alternativa.
        \begin{enumerate}
            \item (   ) Sapato de marca - Cirurgia plástica
            \item (   ) Passagem de ônibus - Remédio
            \item (   ) Restaurantes - Viagens - Carro do ano
            \item (   ) Aluguel da casa - Roupas - Lazer
        \end{enumerate}
    
    \item Escolha a alternativa abaixo que representa a ordem das etapas de um Orçamento Financeiro.
        \begin{enumerate}
            \item (   ) Avaliar, Agrupar, Planejar e Registrar
            \item (   ) Registrar, Planejar, Avaliar, Agrupar
            \item (   ) Planejar, Registrar, Agrupar e Avaliar
            \item (   ) Registrar, Agrupar, Planejar e Avaliar
        \end{enumerate}
    
    \newpage    
    \item Qual o tipo de juros é utilizado para rentabilidade dos investimentos?
        \begin{enumerate}
            \item (   ) Juros Simples
            \item (   ) Juros de Mora
            \item (   ) Juros Compostos
            \item (   ) Juros de Capital Próprio
        \end{enumerate}
        
    \item Considerando que a inflação é a variação mensal dos preços, marque a alternativa que representa o índice oficial que o governo brasileiro utiliza para medir a inflação nacional.
        \begin{enumerate}
            \item (   ) CDI
            \item (   ) SELIC
            \item (   ) CDB
            \item (   ) IPCA
        \end{enumerate}
        
    \item Investimento é aplicação do dinheiro poupado para obtenção de juros ou dividendos. Eles possuem três característica Risco, Rentabilidade e Liquidez. Sabendo disso assinale a alternativa que corresponde a definição de Liquidez.
        \begin{enumerate}
            \item (   ) É a probabilidade de ocorrer perdas no investimento
            \item (   ) É a possibilidade do investimento ser transformado em dinheiro
            \item (   ) É a remuneração recebida pelo investimento
        \end{enumerate}
        
    \item Assinale a alternativa cujo investimento não é garantido pelo FGC.
        \begin{enumerate}
            \item (   ) Tesouro Direto
            \item (   ) LCI/LCA
            \item (   ) CDB
            \item (   ) Poupança
        \end{enumerate}
        
    \item Assinale a características de um Investimento em Renda Variável.
        \begin{enumerate}
            \item (   ) Esse tipo de investimento não possui riscos 
            \item (   ) A rentabilidade desse investimento é garantida
            \item (   ) Esse tipo de investimento possui riscos
            \item (   ) É coberto pelo Fundo Garantidor de Crédito
        \end{enumerate}
\end{enumerate}
