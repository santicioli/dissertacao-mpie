% Seleciona o idioma do documento (conforme pacotes do babel)
%\selectlanguage{english}
\selectlanguage{brazil}

% Retira espaço extra obsoleto entre as frases.
\frenchspacing 

\newpage

% ==============================================
% ELEMENTOS PRÉ-TEXTUAIS
% ==============================================
\pretextual

% ----------------------------------------------
% Capa
% ----------------------------------------------
%\imprimircapa
% Capa personalizada sem o uso de \imprimircapa
\begin{capa} 
   \center
   \ABNTEXchapterfont\large\bfseries{\imprimirinstituicao} 
   \vfill
   %\vspace*{1cm}
   \ABNTEXchapterfont\large\bfseries\textsc{\MakeUppercase{\imprimirautor}}
   \vfill
   \begin{center}
   \ABNTEXchapterfont\Large\bfseries{\MakeUppercase{\imprimirtitulo}}
   \end{center}
   \vfill
   \vspace*{5cm}
   \large\bfseries\MakeTextUppercase{\imprimirlocal} \\
   \large\bfseries\imprimirdata
   \vspace*{1cm}
\end{capa}

% ----------------------------------------------
% Folha de rosto
% ----------------------------------------------
% folha de rosto personalizada sem uso de \imprimirfolhaderosto
\makeatletter
\renewcommand{\folhaderostocontent}{
\begin{center}
  {\ABNTEXchapterfont\large\imprimirautor}
  \vspace*{\fill}%\vspace*{\fill}
  \begin{center}
  \ABNTEXchapterfont\bfseries\Large\imprimirtitulo
  \end{center}
  \vspace*{\fill}
  
  \abntex@ifnotempty{\imprimirpreambulo}{%
    % \hspace{.45\textwidth}
    % \begin{minipage}{.5\textwidth}
    \hspace{.39\textwidth}
    \begin{minipage}{.6\textwidth}
        \SingleSpacing
        \imprimirpreambulo
    \end{minipage}%
    \vspace*{\fill}
  }%
  
  \abntex@ifnotempty{\imprimirorientador}{%
%   \hspace{.45\textwidth}
%   \begin{minipage}{.5\textwidth}
    \hspace{.39\textwidth}
    \begin{minipage}{.6\textwidth}
	    {\imprimirorientadorRotulo~\newline Profº. Dr. \imprimirorientador}%
    \end{minipage}%
  }%
  
  \vspace*{.005\textwidth}
  
  \abntex@ifnotempty{\imprimircoorientador}{%
%   \hspace{.45\textwidth}
%   \begin{minipage}{.5\textwidth}
    \hspace{.39\textwidth}
    \begin{minipage}{.6\textwidth}
    {\imprimircoorientadorRotulo~\newline Profª. Dra. \imprimircoorientador}%
  \end{minipage}%
  }%
  
  \vspace*{\fill}
  %{\abntex@ifnotempty{\imprimirinstituicao}{\imprimirinstituicao\vspace*{\fill}}}

  {\large\imprimirlocal}
  \par
  {\large\imprimirdata}
  \vspace*{1cm}
\end{center}
}
\makeatother

% Folha de rosto (o * indica que haverá a ficha bibliográfica)
\imprimirfolhaderosto*

% ----------------------------------------------
% Inserir a ficha bibliografica catalográfica
% ----------------------------------------------
% Isto é um exemplo de Ficha Catalográfica, ou ``Dados internacionais de catalogação-na-publicação''. Você pode utilizar este modelo como referência. Porem, provavelmente a biblioteca da sua universidade lhe fornecerá um PDF com a ficha catalográfica definitiva após a defesa do trabalho. Quando estiver com o documento, salve-o como PDF no diretório do seu projeto e substitua todo o conteúdo de implementação deste arquivo pelo comando abaixo:

% \begin{fichacatalografica}
%     \includepdf{fig_ficha_catalografica.pdf}
% \end{fichacatalografica}
\begin{fichacatalografica}
	\sffamily
	\vspace*{\fill}					% Posição vertical
  	\begin{center}					% Minipage Centralizado
    	\fbox{
            \begin{minipage}[t]{1,5cm} 
                \vspace{0.5cm} 
                S358j %Algum número que o bibliotecário ira gerar
            \end{minipage}
            
            \begin{minipage}[t]{11cm}	% Largura
            	\small
                \vspace{0.5cm}
            	%\imprimirautor		% ATENCAO - SUBSTITUIR POR %Sobrenome, Nome do autor
                Santicioli, Johnata Souza
            	
            	\hspace{0.5cm} 
                \imprimirtitulo  / \imprimirautor; orientador:  \imprimirorientador; coorientadora: \imprimircoorientador. -- \imprimirlocal: \imprimirdata.\\
            	
            	\hspace{0.5cm}
                \pageref{LastPage} f.\\ % : il. (algumas color.) 
                % ; 30 cm.\\
                
            	
                \hspace{0.5cm}
            	\imprimirtipotrabalho~--~Instituto Federal do Rio Grande do Sul -- Campus Porto Alegre. Programa de Metrado Profissional em Informática na Educação, \imprimirlocal, \imprimirdata. \imprimirorientadorRotulo~Profº. Dr. \imprimirorientador ;
                \imprimircoorientadorRotulo~Profª. Dra. \imprimircoorientador.\\
            	
            	\hspace{0.5cm}
            		1. Alfabetização e Educação Financeira
            		2. Ensino Médio Integrado
            		3. Institutos Federais do Brasil
            		4. Jogos Sérios
            		5. Epistemologia Genética Piagetiana
            		I. Miletto, Evandro Manara.
                    II. Procasko, Josiane Carolina Soares Ramos.
                    III. Título.\\
                
            	\hspace{8.75cm} CDU 621.3 %algum outro numero
            \end{minipage}
        }
        \hspace{0.5cm}
        Dados Internacionais de Catalogação na Publicação (CIP) \\  	
        (Bibliotecário: Nome Sobrenome – CRB 10/1298)
    \end{center}
\end{fichacatalografica}

% ----------------------------------------------
% Inserir errata
% ----------------------------------------------
% \begin{errata}
% Elemento opcional da \citeonline[4.2.1.2]{NBR14724:2011}. Exemplo:

% \vspace{\onelineskip}

% FERRIGNO, C. R. A. \textbf{Tratamento de neoplasias ósseas apendiculares com reimplantação de enxerto ósseo autólogo autoclavado associado ao plasma rico em plaquetas}: estudo crítico na cirurgia de preservação de membro em cães. 2011. 128 f. Tese (Livre-Docência) - Faculdade de Medicina Veterinária e Zootecnia, Universidade de São Paulo, São Paulo, 2011.

% \begin{table}[htb]
% \center
% \footnotesize
% \begin{tabular}{|p{1.4cm}|p{1cm}|p{3cm}|p{3cm}|}
%   \hline
%    \textbf{Folha} & \textbf{Linha}  & \textbf{Onde se lê}  & \textbf{Leia-se}  \\
%     \hline
%     1 & 10 & auto-conclavo & autoconclavo\\
%    \hline
% \end{tabular}
% \end{table}

% \end{errata}

%\endinput

% ----------------------------------------------
% Inserir folha de aprovação
% ----------------------------------------------
% Isto é um exemplo de Folha de aprovação, elemento obrigatório da NBR 14724/2011 (seção 4.2.1.3). Você pode utilizar este modelo até a aprovação do trabalho. Após isso, substitua todo o conteúdo deste arquivo por uma imagem da página assinada pela banca com o comando abaixo:
%
% \includepdf{folhadeaprovacao_final.pdf}
%
%\begin{folhadeaprovacao}
% Gerada conforme instrucoes do PPGEE
%  \begin{center}
%  {\ABNTEXchapterfont\large\imprimirautor}

%  \vspace*{\fill}\vspace*{\fill}
%    \begin{center}
%    	\ABNTEXchapterfont\bfseries\Large\imprimirtitulo
%    \end{center}
%  \vspace*{\fill}

%  \hspace{.45\textwidth}
%    \begin{minipage}{.5\textwidth}
%    	\imprimirpreambulo 
%    \end{minipage}%

%  \vspace*{\fill}
%    \begin{flushleft}
%  	  Aprovado em 20 de Dezembro de 2017. \\
%    \end{flushleft}
%  \vspace*{\fill}
%    BANCA EXAMINADORA:% \imprimirlocal, \today :
%  \end{center}
   
   %\assinatura{\textbf{\imprimirorientador} \\ Orientador}
   %\assinatura{\textbf{\imprimircoorientador} \\ coorientador} 
   
%   \assinatura{Prof. Dr. Nome Sobrenome -- Unisinos \\ Avaliador}
%   \assinatura{Prof. Dr. Nome Sobrenome -- UFRGS \\ Avaliador Externo}
%   \assinatura{Prof. Dr. Nome Sobrenome -- UERGS \\ Avaliador Externo}
   %\assinatura{\textbf{Professor} \\ Convidado 4}
     
 %   \vspace*{\fill} \vspace*{\fill}
 %   \hspace{.4\textwidth}
 %   \begin{minipage}{.5\textwidth}
 %   	\imprimirorientador~(Orientador) \\
 %       \imprimircoorientador~(coorientador)
 %   \end{minipage}%
    
%    \vspace*{\fill}
%    \begin{flushleft}
%    	Visto e permitida a impressão\\
%        \imprimirlocal
%    \end{flushleft}
    
%    \vspace*{\fill}
%    \hspace{.4\textwidth}
%    \begin{minipage}{.5\textwidth}
%    	Profa. Dra. Silvia de Castro Bertagnolli \\
%        Coordenador Mestrado Profissional em Informática na Educação
%    \end{minipage}
%    \begin{center}
%     \vspace*{0.5cm}
%       {\large\imprimirlocal}
%       \par
%       {\large\imprimirdata}
%       \vspace*{1cm}
%   \end{center}
  
%\end{folhadeaprovacao}

% ----------------------------------------------
% Dedicatória
% ----------------------------------------------
% \newpage
% \begin{dedicatoria}
%   \vspace*{\fill}
%   \centering
%   \noindent
%   \textit{A principal meta da educação é criar homens que sejam capazes de fazer coisas novas, não simplesmente repetir o que outras gerações já fizeram. Homens que sejam criadores, inventores, descobridores. A segunda meta da educação é formar mentes que estejam em condições de criticar, verificar e não aceitar tudo que a elas se propõe.\\
%   (Jean Piaget)} 
%   \vspace*{\fill}
% \end{dedicatoria}

% ----------------------------------------------
% Agradecimentos
% ----------------------------------------------
%\begin{agradecimentos}

%    \lipsum[1]
%\end{agradecimentos}

% ----------------------------------------------
% Epígrafe
% ----------------------------------------------
% Importante: O autor da epígrafe deve constar na lista de referências
\newpage
\begin{epigrafe}
    \vspace*{\fill}
	\begin{flushright}
		\textit{Seria uma atitude ingênua esperar que as classes dominantes desenvolvessem uma forma de educação que proporcionasse às classes dominadas perceber as injustiças sociais de maneira crítica.\\
        (Paulo Freire)}
    \end{flushright}
    \begin{flushright}
        \textit{Por que gastais o dinheiro naquilo que não é pão? E o produto do vosso trabalho naquilo que não pode satisfazer? Ouvi-me atentamente, e comei o que é bom, e a vossa alma se deleite com a gordura.\\
        (Isaías 55.2)}
	\end{flushright}
\end{epigrafe}

% ||||||||||||||||||||||||||||||||||||||||||||||
% RESUMOS
% ||||||||||||||||||||||||||||||||||||||||||||||

% ----------------------------------------------
% Resumo em português
% ----------------------------------------------
% Importante: De acordo com a NBR6024 as palavras-chaves devem ser separadas entre si por ponto e devem ter somente a primeira palavra escrita com letra maiúscula
\setlength{\absparsep}{18pt} % ajusta o espaçamento dos parágrafos do resumo
\begin{resumo}
	Uma parcela dos consumidores brasileiros estão endividados excessivamente, inadimplentes ou com restrição ao crédito, esse número tende a aumentar por conta da crise financeira e de saúde causada pela pandemia do COVID-19. Ademais, os jovens adolescentes brasileiros possuem um baixo índice de letramento financeiro, em comparação a outros países que participaram do estudo publicado pela Organização para a Cooperação e Desenvolvimento Econômico. Sem considerar os problemas com pobreza e distribuição de renda do país, esse contexto evidencia o reflexo do baixo letramento e conhecimento financeiro da população brasileira. A inserção desse tema nos currículos acadêmicos representa um grande desafio para as instituições de ensino da Rede Federal de Educação Profissional, Científica e Tecnológica (RFEPCT), em especial, que tem como missão a formação profissional, cidadã e conectada ao mundo contemporâneo. Nesse contexto, o presente projeto tem o objetivo avaliar a efetividade de um curso sobre educação financeira assistido por jogos sérios, como tecnologia educacional principal, para prover educação financeira aos estudantes do ensino médio. A metodologia utilizada no curso é embasada nos estágios do desenvolvimento cognitivo do letramento financeiro, elaborada com apoio da epistemologia genética de Jean Piaget. O processo implica o uso dos jogos Orçamento Consciente e Renda Passiva para simular ambientes, simplificar e apresentar de forma lúdica os temas fundamentais sobre a educação financeira. A presente pesquisa é classificada como aplicada quanto à sua natureza, exploratória quanto aos objetivos e de estudo de caso quanto à sua finalidade e procedimentos, os dados foram coletados por meio de questionários e analisados através da abordagem qualitativa. No intuito de validar a presente pesquisa o curso “Educação Financeira através de Jogos” foi ofertado aos alunos do ensino médio integrado no Instituto Federal Sul-rio-grandense (IFSUL) - Câmpus Gravataí, por intermédio de um projeto de ensino aprovado pela pró-reitoria de ensino da instituição. Os resultados da pesquisa, confirmaram a hipótese que os jogos utilizados favoreceram a aprendizagem dos estudantes, bem como comprovaram que o curso proporcionou aos alunos conhecimentos introdutórios sobre finanças e atendeu as expectativas dos participantes e dos pesquisadores, desta forma ele pode ser expandido para outras unidades de ensino para diminuir essa importante lacuna educacional.
% 	\vspace{\onelineskip}
	
 	\noindent 
	\textbf{Palavras-chaves}: Alfabetização e Educação Financeira; Ensino Médio Integrado; Institutos Federais do Brasil; Jogos Sérios; Epistemologia Genética Piagetiana.
\end{resumo}

% ----------------------------------------------
% Resumo em inglês
% ----------------------------------------------
% Importante: De acordo com a NBR6024 as palavras-chaves devem ser separadas entre si por ponto e devem ter somente a primeira palavra escrita com letra maiúscula
\begin{resumo}[Abstract]
\begin{otherlanguage*}{english}

O resumo em inglês será construído a partir do resumo final do trabalho.    
% 	\vspace{\onelineskip}
 
	\noindent 
	\textbf{Keywords}: Literacy and Financial Education; Integrated High School; Federal Institutes of Brazil; Serious Games; Piagetian Genetic Epistemology.
\end{otherlanguage*}
\end{resumo}

% ----------------------------------------------
% resumo em francês 
% ----------------------------------------------
% Importante: De acordo com a NBR6024 as palavras-chaves devem ser separadas entre si por ponto e devem ter somente a primeira palavra escrita com letra maiúscula
% \begin{resumo}[Résumé]
%  \begin{otherlanguage*}{french}
%     Il s'agit d'un résumé en français.
 
%    \textbf{Mots-clés}: latex. abntex. publication de textes.
%  \end{otherlanguage*}
% \end{resumo}

% ----------------------------------------------
% resumo em espanhol
% ----------------------------------------------
% Importante: De acordo com a NBR6024 as palavras-chaves devem ser separadas entre si por ponto e devem ter somente a primeira palavra escrita com letra maiúscula
% \begin{resumo}[Resumen]
%  \begin{otherlanguage*}{spanish}
%    Este es el resumen en español.
  
%    \textbf{Palabras clave}: latex. abntex. publicación de textos.
%  \end{otherlanguage*}
 
% \end{resumo}

% ----------------------------------------------
% inserir lista de ilustrações (ou figuras)
% ----------------------------------------------
\pdfbookmark[0]{\listfigurename}{lof}
\listoffigures*
\cleardoublepage

% Diferentes tipos de listas podem ser criadas por meio de macros do memoir.

% ----------------------------------------------
% inserir lista de tabelas
% ----------------------------------------------
\pdfbookmark[0]{\listtablename}{lot}
\listoftables*
\cleardoublepage

% ----------------------------------------------
% inserir lista de quadros (ex.: \begin{quadro} \end{quadro})
% ----------------------------------------------
\pdfbookmark[0]{\listofquadrosname}{loq}
\listofquadros*
\cleardoublepage

% ----------------------------------------------
% inserir lista de Graficos (ex.: \begin{grafico} \end{grafico})
% ----------------------------------------------
\pdfbookmark[0]{\listofgraficosname}{logf}
\listofgraficos*
\cleardoublepage

% ----------------------------------------------
% inserir lista de abreviaturas e siglas
% ----------------------------------------------
% Importante: As abreviaturas e siglas devem estar em ordem alfabética
%\begin{siglas}
%  \item[ABNT] Associação Brasileira de Normas Técnicas
%  \item[abnTeX] ABsurdas Normas para TeX
%
\begin{siglas}
    \item[\textbf{AEF}] Associação de Educação Financeira
    \item[\textbf{ANBIMA}] Associação Brasileira das Entidades dos Mercados Financeiros e de Capitais
    \item[\textbf{B3}] Brasil, Bolsa, Balcão
    \item[\textbf{BACEN}] Banco Central do Brasil
    \item[\textbf{BNCC}] Base Nacional Comum Curricular
    \item[\textbf{CDB}] Certificados de Depósito Bancário
    \item[\textbf{CE}] Comissão de Educação
    \item[\textbf{CEFETS}] Centros Federais de Educação Tecnológica
    \item[\textbf{CNC}] Confederação Nacional do Comércio de Bens, Serviços e Turismo 
    \item[\textbf{CNDL}] Confederação Nacional de Dirigentes Lojistas
    \item[\textbf{CONEF}] Comitê Nacional de Educação Financeira
    \item[\textbf{CNSeg}] Confederação Nacional das Empresas de Seguros Gerais, Previdência Privada e Vida, Saúde Suplementar e Capitalização
    \item[\textbf{DCNs}] Diretrizes Curriculares Nacionais Gerais para o ensino básico
    \item[\textbf{ENEF}] Estratégia Nacional de Educação Financeira
    \item[\textbf{ES}] Espírito Santo
    \item[\textbf{FEBRABAN}] Federação Brasileira de Bancos
    \item[\textbf{IFAC}] Instituto Federal do Acre
    \item[\textbf{IFAL}] Instituto Federal de Alagoas
    \item[\textbf{IFAM}] Instituto Federal do Amazonas
    \item[\textbf{IFAP}] Instituto Federal do Amapá
    \item[\textbf{IFB}] Instituto Federal de Brasília
    \item[\textbf{IFBA}] Instituto Federal da Bahia
    \item[\textbf{IFBaiano}] Instituto Federal Bahiano
    \item[\textbf{IFC}] Instituto Federal Catarinense
    \item[\textbf{IFCE}] Instituto Federal do Ceará
    \item[\textbf{IFES}] Instituto Federal do Espírito Santo
    \item[\textbf{IFF}] Instituto Federal Fluminense
    \item[\textbf{IFFarroupilha}] Instituto Federal Farroupilha
    \item[\textbf{IFG}] Instituto Federal de Goiás
    \item[\textbf{IFGoiano}] Instituto Federal Goiano
    \item[\textbf{IFMA}] Instituto Federal do Maranhão
    \item[\textbf{IFMG}] Instituto Federal de Minas Gerais
    \item[\textbf{IFMS}] Instituto Federal do Mato Grosso do Sul
    \item[\textbf{IFMT}] Instituto Federal do Mato Grosso
    \item[\textbf{IFNMG}] Instituto Federal do Norte de Minas Gerais
    \item[\textbf{IFPA}] Instituto Federal do Pará
    \item[\textbf{IFPB}] Instituto Federal da Paraíba
    \item[\textbf{IFPE}] Instituto Federal de Pernambuco
    \item[\textbf{IFPI}] Instituto Federal do Piauí
    \item[\textbf{IFPR}] Instituto Federal do Paraná
    \item[\textbf{IFRJ}] Instituto Federal do Rio de Janeiro
    \item[\textbf{IFRN}] Instituto Federal do Rio Grande do Norte
    \item[\textbf{IFRO}] Instituto Federal de Rondônia
    \item[\textbf{IFRR}] Instituto Federal de Roraima
    \item[\textbf{IFRS}] Instituto Federal do Rio Grande do Sul
    \item[\textbf{IFS}] Instituto Federal de Sergipe
    \item[\textbf{IFSC}] Instituto Federal de Santa Catarina
    \item[\textbf{IFSertão-PE}] Instituto Federal do Sertão Pernambucano
    \item[\textbf{IFSP}] Instituto Federal de São Paulo
    \item[\textbf{IFSudeste MG}] Instituto Federal do Sudeste de Minas Gerais
    \item[\textbf{IFSUL}] Instituto Federal Sul-rio-grandense
    \item[\textbf{IFSul de Minas}] Instituto Federal do Sul de Minas Gerais
    \item[\textbf{IFTM}] Instituto Federal do Triângulo Mineiro
    \item[\textbf{IFTO}] Instituto Federal do Tocantins
    \item[\textbf{LC}] Letras de Crédito
    \item[\textbf{LCA}] Letras de Crédito do Agronegócio
    \item[\textbf{LCI}] Letras de Crédito Imobiliário
    \item[\textbf{MEC}] Ministério da Educação
    \item[\textbf{MEEGA}] \textit {Model for Evaluating Educational Games}
    \item[\textbf{OCDE}] Organização para a Cooperação e Desenvolvimento Econômico
    \item[\textbf{OECD}] \textit {Organisation for Economic Co-operation and Development}
    \item[\textbf{OSCIP}] Organização da Sociedade Civil de Interesse Público
    \item[\textbf{PCN}] Parâmetros Curriculares Nacionais
    \item[\textbf{PCNEM}] Parâmetros Curriculares Nacionais para o Ensino Médio
    \item[\textbf{PISA}] \textit {Program for International Student Assessment}
    \item[\textbf{PL}] Projeto de Lei
    \item[\textbf{RDB}] Recibo de Depósito Bancário
    \item[\textbf{RFEPCT}] Rede Federal de Educação Profissional, Científica e Tecnológica
    \item[\textbf{SCPC}] Serviço Central de Proteção ao Crédito
    \item[\textbf{SFN}] Sistema Financeiro Nacional
    \item[\textbf{SPC}] Serviço de Proteção ao Crédito
    \item[\textbf{SUSEP}] Superintendência de Seguros Privados
    \item[\textbf{TICs}] Tecnologias da Informática e Comunicação
\end{siglas}

%\end{siglas}

%se for usar o arquivo siglas.tex
\pdfbookmark[0]{\listofsiglasname}{losa}
\acrodefplural{AVA}[AVAs]{Ambientes Virtuais de Aprendizagem}
\acrodefplural{IES}[IESs]{Instituições de Ensino Superior}

\chapter*{Lista de Abreviaturas e Siglas}
\begin{acronym}[MPC] % Give the longest label here so that the list is nicely aligned
\acro{AEF}{Associação de Educação Financeira}
\acro{ANBIMA}{Associação Brasileira das Entidades dos Mercados Financeiros e de Capitais}
\acro{B3}{Brasil, Bolsa, Balcão}
\acro{BACEN}{Banco Central do Brasil}
\acro{BNCC}{Base Nacional Comum Curricular}
\acro{CDB}{Certificados de Depósito Bancário}
\acro{CE}{Comissão de Educação}
\acro{CEFETS}{Centros Federais de Educação Tecnológica}
\acro{CNDL}{Confederação Nacional de Dirigentes Lojistas}
\acro{CNSeg}{Confederação Nacional das Empresas de Seguros Gerais, Previdência Privada e Vida, Saúde Suplementar e Capitalização}
\acro{CONEF}{Comitê Nacional de Educação Financeira}
\acro{DCNs}{Diretrizes Curriculares Nacionais Gerais para o ensino básico}
\acro{ENEF}{Estratégia Nacional de Educação Financeira}
\acro{ES}{Espírito Santo}
\acro{FEBRABAN}{Federação Brasileira de Bancos}
\acro{IFAC}{Instituto Federal do Acre}
\acro{IFAL}{Instituto Federal de Alagoas}
\acro{IFAM}{Instituto Federal do Amazonas}
\acro{IFAP}{Instituto Federal do Amapá}
\acro{IFB}{Instituto Federal de Brasília}
\acro{IFBA}{Instituto Federal da Bahia}
\acro{IFBaiano}{Instituto Federal Bahiano}
\acro{IFC}{Instituto Federal Catarinense}
\acro{IFCE}{Instituto Federal do Ceará}
\acro{IFES}{Instituto Federal do Espírito Santo}
\acro{IFF}{Instituto Federal Fluminense}
\acro{IFFarroupilha}{Instituto Federal Farroupilha}
\acro{IFG}{Instituto Federal de Goiás}
\acro{IFGoiano}{Instituto Federal Goiano}
\acro{IFMA}{Instituto Federal do Maranhão}
\acro{IFMG}{Instituto Federal de Minas Gerais}
\acro{IFMS}{Instituto Federal do Mato Grosso do Sul}
\acro{IFMT}{Instituto Federal do Mato Grosso}
\acro{IFNMG}{Instituto Federal do Norte de Minas Gerais}
\acro{IFPA}{Instituto Federal do Pará}
\acro{IFPB}{Instituto Federal da Paraíba}
\acro{IFPE}{Instituto Federal de Pernambuco}
\acro{IFPI}{Instituto Federal do Piauí}
\acro{IFPR}{Instituto Federal do Paraná}
\acro{IFRJ}{Instituto Federal do Rio de Janeiro}
\acro{IFRN}{Instituto Federal do Rio Grande do Norte}
\acro{IFRO}{Instituto Federal de Rondônia}
\acro{IFRR}{Instituto Federal de Roraima}
\acro{IFRS}{Instituto Federal do Rio Grande so Sul}
\acro{IFS}{Instituto Federal de Sergipe}
\acro{IFSC}{Instituto Federal de Santa Catarina}
\acro{IFSertão-PE}{Instituto Federal do Sertão Pernambucano}
\acro{IFSP}{Instituto Federal de São Paulo}
\acro{IFSudeste MG}{Instituto Federal do Sudeste de Minas Gerais}
\acro{IFSUL}{Instituto Federal Sul-rio-grandense}
\acro{IFSul de Minas}{Instituto Federal do Sul de Minas Gerais}
\acro{IFTM}{Instituto Federal do Triângulo Mineiro}
\acro{IFTO}{Instituto Federal do Tocantins}
\acro{LC}{Letras de Crédito}
\acro{LCA}{Letras de Crédito do Agronegócio}
\acro{LCI}{Letras de Crédito Imobiliário}
\acro{MEC}{Ministério da Educação}
\acro{MEEGA}{\textit {Model for Evaluating Educational Games}}
\acro{OCDE}{Organização para a Cooperação e Desenvolvimento Econômico}
\acro{OECD}{\textit {Organisation for Economic Co-operation and Development}}
\acro{OSCIP}{Organização da Sociedade Civil de Interesse Público}
\acro{PCN}{Parâmetros Curriculares Nacionais}
\acro{PCNEM}{Parâmetros Curriculares Nacionais para o Ensino Médio}
\acro{PISA}{\textit {Program for International Student Assessment}}
\acro{PL}{Projeto de Lei}
\acro{RDB}{Recibo de Depósito Bancário}
\acro{RFEPCT}{Rede Federal de Educação Profissional, Científica e Tecnológica}
\acro{SAC}{Sistema de Amortização Constante}
\acro{SCPC} {Serviço Central de Proteção ao Crédito}
\acro{SPC}{Serviço de Proteção ao Crédito}
\acro{SUSEP}{Superintendência de Seguros Privados}
\acro{TICs}{Tecnologias da Informática e Comunicação}
\end{acronym}
\cleardoublepage

% ----------------------------------------------
% inserir lista de símbolos
% ----------------------------------------------
% Importante: Os símbolos devem estar na ordem de aparecimento no texto.

%\begin{simbolos}
%  \item[$ \Gamma $] Letra grega Gama
%  \item[$ \Lambda $] Lambda
%  \item[$ \zeta $] Letra grega minúscula zeta
%  \item[$ \in $] Pertence
%\end{simbolos}

% ----------------------------------------------
% inserir o sumário
% ----------------------------------------------
\pdfbookmark[0]{\contentsname}{toc}
\tableofcontents*
\cleardoublepage
