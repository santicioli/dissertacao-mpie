A portion of Brazilian consumers are excessively indebted, in default or with restricted credit, this number tends to increase due to the financial and health crisis caused by the pandemic of COVID-19. In addition, young Brazilian adolescents have a low rate of financial literacy, compared to other countries that participated in the study published by the Organization for Economic Cooperation and Development. Without considering the problems with poverty and income distribution in the country, this context highlights the reflection of the low literacy and financial knowledge of the Brazilian population. The insertion of this theme in the academic curricula represents a great challenge for the educational institutions of the Rede Federal de Educação Profissional, Científica e Tecnológica (RFEPCT), in particular, whose mission is professional, citizen and connected education to the contemporary world. In this context, the present project aims to evaluate the effectiveness of a course on financial education assisted by serious games, as the main educational technology, to provide financial education to high school students. The methodology used in the course is based on the stages of cognitive development of financial literacy, developed with the support of Jean Piaget's genetic epistemology. The process implies the use of the Orçamento Consciente and Renda Passiva games to simulate environments, simplify and playfully present the fundamental themes of financial education. The present research is classified as applied as to its nature, exploratory as to the objectives and as a case study as to its purpose and procedures, the data were collected through questionnaires and analyzed through the qualitative approach. In order to validate this research, the “Financial Education through Games” course was offered to high school students integrated at the Instituto Federal Sul-rio-grandense (IFSUL) - Câmpus Gravataí, through a teaching project approved by the teaching rectory of the institution. The results of the research confirmed the hypothesis that the games used favored student learning, as well as proved that the course provided students with introductory knowledge about finance and met the expectations of participants and researchers, in this way it can be expanded to other units to reduce this important educational gap.