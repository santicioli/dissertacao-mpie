A considerable portion of the Brazilian population is unable to meet it's financial commitments, therefore the number of defaulters (with arrears) has grown in recent years. It is estimated that approximately 40 {\%} of the adult population, in 2018, had their name enrolled in credit protection services for not paying debts acquired with financial institutions (SPC - CNDL, \citeyear{spc2018}). This number tends to increase, because, according to a survey conducted by the National Confederation of Trade in Riches, Services and Tourism, the injection of credit due to the COVID-19 pandemic made the percentage of indebted families (with debts beyond financial capacity) reach 67.1 {\%} and reach a historical record in June 2020 \cite{cnc2020}. In addition, young Brazilian adolescents have the lowest index of knowledge about finance, compared to other member countries and potential members of the Organization for Economic Cooperation and Development \cite{oecd2017}, which participated in the study on financial literacy. Without considering the problems with poverty and income distribution in the country, this context highlights the reflection of the low financial literacy and the lack of financial education of the Brazilian population in the national social and economic scenario. The insertion of this theme in the academic curricula represents a great challenge for the educational institutions of the Federal Network of Professional, Scientific and Technological Education (RFEPCT), in particular, whose mission is to prepare a professional citizen and connected to the world of work contemporary \cite{brasil2008}. In view of this, this project aims to propose a course attended by serious games as a methodology to promote literacy and financial education to students of the RFEPCT integrated high school. Through a methodology based on Piaget's constructivist educational theory in his studies on genetic epistemology, the process implies the use of serious games to simplify and playfully present fundamental themes of literacy and financial education such as the financial market, financial behavior and personal finances, which have a relative degree of complexity and dynamism. The present research is classified as applied as to it's nature, exploratory as to the objectives and case study as to it's purpose and procedures, in the analysis of the collected data a qualitative approach methodology was used. Methodology assisted by serious games, proposed in the present study, was applied in the Course of Introduction to Financial Education for Integrated High School at the Instituto Federal Sul-rio-grandense (IFSUL) - Câmpus Gravataí, through a teaching project. As a final result, it is expected to develop an efficient methodology in a course format to impact on the learning process of literacy and financial education of the integrated high school students of RFEPCT, which can be expanded to other units and networks to contribute to reducing this important gap educational.