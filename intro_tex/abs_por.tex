Uma parcela considerável da população brasileira não consegue cumprir com seus compromissos financeiros, por isso o número de inadimplentes (com dívidas atrasadas) tem crescido nos últimos anos. Estima-se que aproximadamente 40{\%} da população adulta, em 2018, possuía o nome inscrito em serviços de proteção ao crédito por não pagar dívidas adquiridas com instituições financeiras (SPC - CNDL, \citeyear{spc2018}). Este número tende a aumentar, pois, de acordo com a pesquisa realizada pela Confederação Nacional do Comércio de Bens, Serviços e Turismo, a injeção de crédito por conta da pandemia do COVID-19 fez o percentual de famílias endividadas (com dívidas além da capacidade financeira) chegar à 67,1{\%} e alcançar recorde histórico em junho de 2020 \cite{cnc2020}. Ademais, os jovens adolescentes brasileiros possuem o menor índice de conhecimento sobre finanças, em comparação a outros países membros e potenciais membros da Organização para a Cooperação e Desenvolvimento Econômico \cite{oecd2017}, que participaram do estudo sobre alfabetização financeira. Sem considerar os problemas com pobreza e distribuição de renda do país, este contexto evidencia o reflexo da baixa alfabetização financeira e da falta de educação financeira da população brasileira no cenário social e econômico nacional. A inserção deste tema nos currículos acadêmicos, representa um grande desafio para as instituições de ensino da Rede Federal de Educação Profissional, Científica e Tecnológica (RFEPCT), em especial, que tem como missão a formação de um profissional cidadão e conectado ao mundo do trabalho contemporâneo \cite{brasil2008}. Em vista disso, este projeto tem o objetivo propor um curso assistido por jogos sérios como metodologia para promover alfabetização e educação financeira aos estudantes do ensino médio integrado da RFEPCT. Por meio de uma metodologia embasada pela teoria educacional construtivista de Piaget em seus estudos sobre a epistemologia genética, o processo implica o uso de jogos sérios para simplificar e apresentar de forma lúdica temas fundamentais da alfabetização e educação financeira como o mercado financeiro, comportamento financeiro e finanças pessoais, que possuem um relativo grau de complexidade e dinamismo. A presente pesquisa é classificada como aplicada quanto à sua natureza, exploratória quanto aos objetivos e de estudo de caso quanto à sua finalidade e procedimentos, na análise dos dados coletados utilizou-se uma metodologia de abordagem qualitativa. Aplicou-se metodologia assistida por jogos sérios, proposta no presente estudo, no Curso de Introdução à Educação Financeira para o Ensino Médio Integrado no Instituto Federal Sul-rio-grandense (IFSUL) - Câmpus Gravataí, por intermédio de um projeto de ensino. Como resultado final, espera-se elaborar uma metodologia eficiente em formato de curso para impactar no processo de aprendizagem da alfabetização e educação financeira dos alunos do ensino médio integrado da RFEPCT, podendo ser expandido para outras unidades e redes para contribuir na diminuição dessa importante lacuna educacional.