Uma parcela dos consumidores brasileiros estão endividados excessivamente, inadimplentes ou com restrição ao crédito, esse número tende a aumentar por conta da crise financeira e de saúde causada pela pandemia do COVID-19. Ademais, os jovens adolescentes brasileiros possuem um baixo índice de letramento financeiro, em comparação a outros países que participaram do estudo publicado pela Organização para a Cooperação e Desenvolvimento Econômico. Sem considerar os problemas com pobreza e distribuição de renda do país, esse contexto evidencia o reflexo do baixo letramento e conhecimento financeiro da população brasileira. A inserção desse tema nos currículos acadêmicos representa um grande desafio para as instituições de ensino da Rede Federal de Educação Profissional, Científica e Tecnológica (RFEPCT), em especial, que tem como missão a formação profissional, cidadã e conectada ao mundo contemporâneo. Nesse contexto, o presente projeto tem o objetivo avaliar a efetividade de um curso sobre educação financeira assistido por jogos sérios, como tecnologia educacional principal, para prover educação financeira aos estudantes do ensino médio. A metodologia utilizada no curso é embasada nos estágios do desenvolvimento cognitivo do letramento financeiro, elaborada com apoio da epistemologia genética de Jean Piaget. O processo implica o uso dos jogos Orçamento Consciente e Renda Passiva para simular ambientes, simplificar e apresentar de forma lúdica os temas fundamentais sobre a educação financeira. A presente pesquisa é classificada como aplicada quanto à sua natureza, exploratória quanto aos objetivos e de estudo de caso quanto à sua finalidade e procedimentos, os dados foram coletados por meio de questionários e analisados através da abordagem qualitativa. No intuito de validar a presente pesquisa o curso “Educação Financeira através de Jogos” foi ofertado aos alunos do ensino médio integrado no Instituto Federal Sul-rio-grandense (IFSUL) - Câmpus Gravataí, por intermédio de um projeto de ensino aprovado pela pró-reitoria de ensino da instituição. Os resultados da pesquisa, confirmaram a hipótese que os jogos utilizados favoreceram a aprendizagem dos estudantes, bem como comprovaram que o curso proporcionou aos alunos conhecimentos introdutórios sobre finanças e atendeu as expectativas dos participantes e dos pesquisadores, desta forma ele pode ser expandido para outras unidades de ensino para diminuir essa importante lacuna educacional.