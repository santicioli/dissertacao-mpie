Uma parcela considerável da população brasileira não consegue cumprir com seus compromissos financeiros. Estima-se que aproximadamente 40\% da população adulta possui o nome inscrito nos serviços de proteção ao crédito. Ademais os jovens e adolescentes brasileiros possuem o menor índice de conhecimento sobre finanças de acordo com a OCDE, fica evidente o reflexo da baixa alfabetização e educação financeira no Brasil no cenário social e econômico do país. A inserção deste tema nos currículos acadêmicos, representa um grande desafio para as instituições de ensino da Rede Federal de Educação Profissional, Científica e Tecnológica (RFEPCT), em especial, que tem como missão a formação de um profissional cidadão e conectado ao mundo do trabalho contemporâneo. Em vista disso, este projeto tem o objetivo propor uma metodologia assistida por jogos para promover a Alfabetização e a Educação Financeira aos estudantes do ensino médio integrado da RFEPCT. O processo implica do uso de jogos educacionais embasada pela teoria educacional construtivista de Piaget para simplificar e apresentar de forma lúdica temas fundamentais da Alfabetização e Educação Financeira como mercado financeiro, comportamento e finanças pessoais que possuem um relativo grau de complexidade. Para o desenvolvimento desse projeto, será utilizada metodologia com abordagem qualitativa na análise dos dados, classificada como aplicada quanto à sua natureza, exploratória quanto aos objetivos e de estudo de caso quanto à sua finalidade e procedimentos. Aplicar-se-á a metodologia proposta e a pesquisa no Curso de Introdução à ALfabetização e Educação Financeira para o Ensino Médio Integrado no Instituto Federal Sul-rio-grandense (IFSUL) - Câmpus Gravataí por meio de um projeto de ensino. Como resultado final, espera-se elaborar uma metodologia eficiente em formato de curso para impactar no processo de aprendizagem da Alfabetização Financeira dos alunos do Ensino Médio Integrado da RFEPCT, podendo ser expandido para outras redes para contribuir na diminuição dessa importante lacuna educacional.